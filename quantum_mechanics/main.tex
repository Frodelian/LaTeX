\documentclass[
a4paper, 
fontsize=10pt, 
twoside=false, 

]{kaobook}
\usepackage[T2A]{fontenc}
\usepackage[utf8]{inputenc}
\usepackage[russian]{babel}
\usepackage[english=british]{csquotes}


% Пакеты Американского математического общества 
\usepackage{amsmath, amsfonts, amssymb, amsthm, mathtools}




\setmainfont{Times New Roman}

\title{\textbf{\textsc{Квантовая механика}}}
\author{}
\date{} %Для того чтобы не было даты
\linespread{1.3} % Отступ сверху для титульника

\begin{document}
	
	\maketitle % Титульник
	\thispagestyle{empty} % Отсутствие номера страницы на титульной странице
	\clearpage % Делаем пустой титульник чтобы перенести оглавление
	\setcounter{page}{1} % Сброс счетчика страниц
		
	\tableofcontents
	\clearpage
	\section{Место квантовой теории в современной картине мира}
	\subsection{Вглубь вещества}
	                 
    Вещества состоят из \textit{молекул} и \textit{атомов}. \textit{Молекулы} состоят из \textit{атомов}.
    \par Каждый отдельный \textit{атом} состоит из \textit{ядра} и некоторого количества \textit{электронов} ($\boldsymbol{e}$ - электрический заряд равен -1 в единицах элементарного заряда).
    \par \textit{Атомные ядра} состоят из протонов ($\boldsymbol{p}$ - заряд +1) и \textit{нейтронов} ($\boldsymbol{n}$ - заряд 0), которое "склеены" между собой с помощью \textit{глюонов} (квантов сильного взаимодействия).
    \par \textit{Протоны} и \textit{нейтроны} состоят из $\boldsymbol{u}$ (заряд $\frac{2}{3}$) и $\boldsymbol{d}$ (заряд -$\frac{1}{3}$) кварков($\boldsymbol{p}$ = $\boldsymbol{u}$$\boldsymbol{u}$$\boldsymbol{d}$, $\boldsymbol{n}$ = $\boldsymbol{u}$$\boldsymbol{d}$$\boldsymbol{d}$). 
    \par \textit{Кварки} и \textit{электроны} считаются \textbf{истинно элементарными частицами}: они ни из чего не состоят, но могут превращаться\footnote{Превращение одной частицы в несколько других могут называть распадом, но это не значит, что продукты распада присутствовали внутри исходной частицы, правильнее считать, что продукты распада возникли в момент превращения} в другие частицы.
    \subsubsection {Частицы и поля}
    \par Частицы и даже истинно элементарные частицы могут иметь \textit{внутренние степени свободы}, которые не связаны с движением частицы как целого(движение частицы в пространстве как единого объекта).\textbf{К внутренним степеням свободы можно отнести} различные \textit{заряды}\footnote{Зарядами обычно называют сохраняющиеся величины, не зависящие от системы отчёта, например электрический заряд - это заряд. Энергия, импульс и момент импульса сохраняются, но зависят от системы отчёта и зарядами не считаются}, а также собственный момент импульса - \textit{спин}. Состояние внутренних степеней свободы частицы может также называться \textit{поляризацией}.
    
	\clearpage
	\section{Основная часть}
	Тут будет основная часть текста.
	\clearpage
	\subsection{Подраздел}
	Тут будет текст подраздела.
	\clearpage
	\section{Заключение}
	Здесь будет заключение.
	
\end{document}