\documentclass[
a4paper, 
fontsize=10pt, 
twoside=false, 

]{kaobook}
\usepackage[T2A]{fontenc}
\usepackage[utf8]{inputenc}
\usepackage[russian]{babel}
\usepackage[english=british]{csquotes}


% Пакеты Американского математического общества 
\usepackage{amsmath, amsfonts, amssymb, amsthm, mathtools}




\setmainfont{Times New Roman}

\title{\textbf{\textsc{Инструкция по выживанию для системного администратора}}}
\author{}
\date{} % Для того чтобы не было даты
\linespread{1.3} % Отступ сверху для титульника

\hypersetup{hidelinks} % Чёрные ссылки
% Цветные ссылки
%\hypersetup{
    %colorlinks=true,
    %linkcolor=blue,
    %filecolor=magenta,
    %urlcolor=cyan,
    %citecolor=green,
    %pdfpagemode=FullScreen,
%}

\begin{document}
	
%%%%%%%%%%%%%%%%%%%%%%%%%%%%%%%%%%%%%%%%%%%%%%%%%%%%%%%%%%%%%%%%%%%%%%%%%%%%%%%%%%
	
	\maketitle % Титульник
	\thispagestyle{empty} % Отсутствие номера страницы на титульной странице
	\clearpage % Делаем пустой титульник чтобы перенести оглавление
	\setcounter{page}{1} % Сброс счетчика страниц
	\thispagestyle{empty}
	\tableofcontents
	\clearpage
	
%%%%%%%%%%%%%%%%%%%%%%%%%%%%%%%%%%%%%%%%%%%%%%%%%%%%%%%%%%%%%%%%%%%%%%%%%%%%%%%%%%
%%%%%%%%%%%%%%%%%%%%%%%%%%%%%%%%%%%%%%%%%%%%%%%%%%%%%%%%%%%%%%%%%%%%%%%%%%%%%%%%%%	

	\section{Что нужно понимать при трудоустройстве?}
	
%%%%%%%%%%%%%%%%%%%%%%%%%%%%%%%%%%%%%%%%%%%%%%%%%%%%%%%%%%%%%%%%%%%%%%%%%%%%%%%%%%	

	\subsection{Узнать о целевых показателях при трудоустройстве}
	
	Обычно работодатель, нанимая системного администратора на работу, ожидает, что с ним-то теперь все будет работать «хорошо». Но вот что он вкладывает в это понятие зачастую покрыто мраком и в большинстве случаев заканчивается тем, что системный администратор из ИТ-специалиста превращается в мальчика для битья в каждом случае, когда становится не так хорошо, как этого хотелось.
	Еще устраиваясь на работу договаривайтесь с вашим работодателем о том, какие ожидания у него от вашей работы: начиная от близких к вашему телу вещей (ваш график работы, доступность в выходные дни и вечерами, как у вас будет проходить отпуск), заканчивая вопросами бизнеса (какие ИТ-сервисы критичны, какой их максимальный простой допустим, какая максимальная потеря данных допустима и т.д.). Все эти параметры должны быть измеримы (работа с 9 до 18, но по телефону доступен до 22, Интернет должен, в случае чего, чиниться за 2 часа и т.д.). Получив эту информацию, не поленитесь закрепить ее письменно – это очень сильно поможет вам в дальнейшем, а пока, заступив на работу:
	
	\subsection{Делайте резервное копирование и согласовывайте схему резервного копирования с руководством}
	
	Настройте автоматическое создание резервных копий и регулярно проверяйте корректность их создания. Еще раз согласуйте схему резервного копирования с руководством и сделайте это письменно – в дальнейшем это поможет вам ответить на вопросы «почему у нас база копировалась всего раз в день, а не каждый час». В случае если руководство хочет большей глубины хранения или частоты создания резервных копий – обозначьте необходимые инвестиции в оборудование. И да, не храните резервные копии на тех же дисковых массивах, что и основные данные, ибо:
	
	\subsection{Оценивайте риски и не полагайтесь на надежность железа, программного обеспечения и каналов связи}
	
    Именно этот дорогущий брендовый (к тому же, единственный) сервер откажет в самый неподходящий момент и лишит вас сна на пару суток, заставив вас судорожно придумывать, что можно сделать в сложившейся ситуации. Проведите небольшой анализ эксплуатационных рисков: что вы (организация) будете делать, если сломается сервер/компьютер/принтер, откажет канал Интернет, возникнет какая-то ошибка в ПО или пожар в серверной? Сколько времени уйдет на восстановление работоспособности? Соответствует ли это время ожиданиям бизнеса (см. пункт 1)? Что можно сделать, чтобы уменьшить время простоя и негативные последствия? После этого опять же задокументируйте ваши выводы и представьте их на суд руководству:
    
    \subsection{Согласовывайте эксплуатационные риски и планы аварийного восстановления с руководством}
	
	Хуже админа, который намеренно вредит, есть только админы, которые вредят из лучших соображений. Вы именно сегодня в последний день сдачи отчетности решили продуть сервер от пыли, вы решили обновить систему на серверах именно посреди проектной сессии, а миграцию почты на новую почтовую систему вы делаете именно в тот момент, когда вся компания ждет именно то самое ценное и важное письмо. Список можно продолжать, но смысл один – всегда доносите до руководства ваши планы, чтобы оно хотя бы понимало необходимость тех небольших неудобств, без которых иногда никак не обойтись. Если для проведения каких-то изменений вам не хватает опыта, то не стесняйтесь обратиться за советом к профессионалам:
	
	\subsection{Четко определяйте зону вашей компетентности и некомпетентности и не бойтесь обсуждать этот вопрос с руководством}
	
	Когда то давно я думал, что профессионал – это тот, кто знает и умеет все. С возрастом пришло понимание, что профессионал – это тот, кто знает границы своей компетентности и некомпетентности. Есть профессиональные сетевые администраторы, профессиональные инженера по системам хранения данных, профессиональные специалисты техподдержки и т.д., но нет людей, которые знают все обо всем. Если в обслуживаемой вами компании зоопарк из технологий, то попытка обслуживать все это единолично (пусть даже в связи и с небольшими объемами) вызывает большое уважение у коллег по цеху, но вряд ли когда-либо будет по достоинству оценено работодателем, а вот спрашивать с вас будут за все это по полной мере. Обозначьте руководству, с какими системами вы знакомы в полном объеме, а где ваших знаний может быть недостаточно – пусть они решают оставлять ли все как есть, нанимать еще одного специалиста или использовать услуги экспертной технической поддержки от внешней ИТ-компании. В целом возьмите за правило и:
	
	\subsection{Всегда разделяйте ответственность за принимаемые решения, особенно если оно принято вами не единолично}
	
	Перед вами стоит задача купить оборудование, вы выбрали модель, запросили счет, но в последний момент отдел закупок наложил свое вето – на яндексмаркете есть дешевле! Ок, не проблема, но в этом случае пусть отдел закупок и несет ответственность за закупку данного оборудования. Я знаю пару компаний, где отдел закупок уже 2 года ждет доставки закупленных «по самой низкой цене» ноутбуков. Точно также и со всеми остальными решениями – лучше всего утверждать их у руководителя, заранее показав возможные альтернативы, и получив от него подтверждение о правильности принятого решения. И да, на всякий случай:
	
	\subsection{Согласовывайте все решения письменно}
	
	Приучайте всех, что все решения согласовываются письменно – простое электронное письмо поможет вам, когда начнутся разбирательства (не «если», а именно «когда»). Согласовывать надо все без исключения решения, чтобы у людей не было защитной реакции, когда вы вдруг ни с того ни с сего присылаете письмо-согласование, хотя до этого все решения принимали устно. Очень частая ситуация, когда ваши коллеги игнорируют письменные согласования – в этом случае согласовывайте с ними решение устно, а после разговора шлите им письмо вида «Уважаемый Пал Палыч, по результатам встречи высылаю перечень согласованных с вами решений:…» — в случае разбирательств к Пал Палычу как минимум будут вопросы почему он еще тогда никак не отреагировал на ваше письмо. В целом письменные согласования это одна из процедур, которые вам необходимо будет ввести для того, чтобы качественно справляться со своей работой:
	
	\subsection{Создайте понятные и удобные всем правила взаимодействия с вами. Делайте все для того, чтобы пользователи не испытывали беспокойства}
	
	В небольших организациях есть очень распространенный способ решения проблем в работе компьютера: подойди к системному администратору, взять его за руку, привести к своему компьютеру, показать проблему, не отпускать пока проблема не будет решена. Естественно ни о какой вдумчивой, системной, сосредоточенной работе по текущему сопровождению и настройке серверов при таком подходе пользователей речи не идет. Но обычно к такому способу решения проблем пользователи приходят эмпирическим путем, т.к. если просто сказать, что у тебя проблема, то админ придет непонятно когда, а придя еще не факт что сделает, а работать надо и время не ждет и это все заставляет волноваться пользователей и, в конечном итоге, заставляет их идти на такие смешные, на первый взгляд, поступки. Не заставляйте ваших пользователей волноваться: обозначайте им сроки, когда Вы сможете им помочь, обозначайте сроки, когда проблема будет устранена, и в случаях, когда у вас в ближайшие дни планируется много серьезной работы заранее пишите им письма, что в этим дни у вас приемные часы со стольки-то до стольки-то (конечно же, с указанием причины). Попросите их подавать несрочные задачи письменно и обещайте отрабатывать их в определенные временные рамки. И самое главное: выдерживайте Ваши обещания – доверие и любовь пользователей Вам будет обеспечена.
	Ну и после всего вышеперечисленного не забывайте делать следующее:
	
	\subsection{Регулярно проверяйте созданные резервные копии}
	
	Все вы знаете старый анекдот, что есть два типа сисадминов: которые еще не делают резервные копии, и те, которые уже делают. Я бы этот анекдот изменил, добавив, что есть третий тип – который еще и проверяет созданные резервные копии. Поймите меня правильно, но в нашем деле потерять данные – это верх непрофессионализма. Сколько бы суток в день вы не были заняты на работе, всегда выделяйте время на то, чтобы проверить, что резервные копии данных у вас создаются, создаются корректно и что из них можно эти самые данные восстановить.

    \subsection{Повторяйте цикл: «снятие потребностей – анализ рисков – согласование – приведение в соответствие» хотя бы раз в год}
    
    Малый бизнес меняется очень быстро: еще вчера в компании выписывали 3 счета в неделю, а сегодня их выписывают 20 штук в минуту. Еще вчера всем хватало только ежедневной резервной копии данных, а сегодня потеря информации за 5 минут уже критичная для бизнеса. Ваша задача всегда быть в авангарде всех изменений в бизнесе и опережать их. Ну и в качестве послесловия хочется добавить, что в любой организации формирование структурного подразделения начинается с его руководителя. Так что если вы в вашей компании единственный понимающий в ИТ человек, то, уверяю вас, вы уже ИТ-директор, вне зависимости от того, что у вас написано в трудовой, спрашивать с вас будут как с руководителя, а не как с исполнителя. Надеюсь, что приведенная выше инструкция позволит вам немного по другому взглянуть на вашу повседневную работу, легче влиться в новую для вас роль и сделает вашу работу максимально спокойной и предсказуемой.

	\clearpage

%%%%%%%%%%%%%%%%%%%%%%%%%%%%%%%%%%%%%%%%%%%%%%%%%%%%%%%%%%%%%%%%%%%%%%%%%%%%%%%%%%   
 
    
	\begin{figure}[h]
		\centering
		\includegraphics[height=8cm]{img/11} 
		\captionsetup{font=footnotesize}
		\caption{Зонная диаграмма(а) и распределение электронов по энергетическим уровням (б) полупроводника с акцепторными примесями} 
	\end{figure}

%%%%%%%%%%%%%%%%%%%%%%%%%%%%%%%%%%%%%%%%%%%%%%%%%%%%%%%%%%%%%%%%%%%%%%%%%%%%%%%%%%
	
\end{document}