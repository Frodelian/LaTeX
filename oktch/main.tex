\documentclass[
a4paper, 
fontsize=10pt, 
twoside=false, 

]{kaobook}
\usepackage[T2A]{fontenc}
\usepackage[utf8]{inputenc}
\usepackage[russian]{babel}
\usepackage[english=british]{csquotes}


% Пакеты Американского математического общества 
\usepackage{amsmath, amsfonts, amssymb, amsthm, mathtools}




\setmainfont{Times New Roman}

\title{\textbf{\textsc{Основы комбинаторики и теории чисел}}}
\author{Боднар Олег Леонидович}
\date{} % Для того чтобы не было даты
\linespread{1.3} % Отступ сверху для титульника

\hypersetup{hidelinks} % Чёрные ссылки
% Цветные ссылки
%\hypersetup{
    %colorlinks=true,
    %linkcolor=blue,
    %filecolor=magenta,
    %urlcolor=cyan,
    %citecolor=green,
    %pdfpagemode=FullScreen,
%}

\pagestyle{fancy} % Устанавливаем стиль страницы
\fancyhf{}
\lhead{\small\hyperref[sec:toc]{\leftmark}} % Тема слева
\rhead{\thepage} % Страница справа
\renewcommand{\headrulewidth}{1pt} % Убрать линию в контитулах

\begin{document}
	
	\maketitle % Титульник
	\thispagestyle{empty} % Отсутствие номера страницы на титульной странице
	\clearpage % Делаем пустой титульник чтобы перенести оглавление
	\setcounter{page}{1} % Сброс счетчика страниц
	\thispagestyle{empty}
	\tableofcontents\label{sec:toc}
	\clearpage

	\section{Элементы теории множеств}
	
	\textbf{Теория множеств} - это раздел математики, в котором изучают общие свойства множеств. Данная теория лежит в основе большинства математических дисциплин, в том числе математического анализа, геометрии и теории вероятности.
	
	\subsection{Основные термины и определения}
	
	\begin{definition}
	\textit{Множеством} называется произвольный набор (совокупность, класс, семейство) каких либо объектов. Объекты, входящие во множество, называются его \textit{элементами}. Если объект $x$ является элементом множества $A$, то говорят, что $x$ \textit{принадлежит} $A$, и пишут $x$ $\in$ $A$.
	\end{definition}
	
	Три базовых понятия: множество, элемент и принадлежность. Множество - это то, чему принадлежат элементы.  Элементы - это то, что принадлежит множеству. А принадлежность - это то, как относится элемент к множеству. 

    \par Два способа записи множеств: 
    \\
    1) Перечисление   
    \\
    Например $A = \{6, 28, 496\}$  или $N = \{0,1,2,3,4,\ldots\}$. При этом каждый элемент должен встречаться в перечислении ровно один раз: запись $\{1,1,2,3\}$ нужно признать либо не имеющей смысла, либо эквивалентной $\{1,2,3\}$. Иногда рассматривают \textit{мультимножества}, в которые каждый элемент может входить несколько раз. При записи множеств не важен порядок, в котором идут элементы. 
    Если множество содержит конечное число элементов, оно называется \textit{конечным}, в противном случае - \textit{бесконечным}
    \\
    2) Set builder notation (формулировка определяющего свойства). 
    \\
    Например, $\{x \ | \ x > 0\}$ - множество всех положительных  $x$. Можно также явно указать какому объемлющему множеству  все элементы. Например, $\{x\in \mathbb{R} \ | \ x>0\}$ - множество всех положительных действительных чисел. Иногда вместо черты $( \ | \ )$ используют двоеточие $( \ : \ )$, особенно когда черта уже встречается в формуле.  Например, запись $\{x\in \mathbb{R} : |x| < 1\}$. Слева от черты могут стоять более сложные выражения. Например, $\{(a,b,c) \ | \ a^2 + b^2  = c^2, \ a,b,c \in  \mathbb{N}, \ a,b,c>0\}$ обозначает множество всех пифагоровых троек, $\{a^2 \ | \ a \in \mathbb{N}\}$ обозначает множество всех полных квадратов.
    
    \begin{definition}
    Множество $A$ является \textit{подмножеством} множества $B$ (или "лежит в множестве $B$", или "включено в $B$"), если любой элемент множества $A$ также принадлежит множеству $B$. Обозначение: $A \subseteq B$. 
    \\
    Множества $A$ и $B$ \textit{равны} если $A \subset B$ и $B \subset A$.  Обозначение: $A = B$. 
    \\
    Если $A \subset B$ , но $A \ne B$, то $A$ называют собственным или строгим подмножеством. Обозначение $A \subsetneq  B$.
    \end{definition}
    
	\clearpage

	\section{Заключение}
	Здесь будет заключение.
	
	
	
\end{document}