\documentclass[
a4paper, 
fontsize=10pt, 
twoside=false, 

]{kaobook}
\usepackage[T2A]{fontenc}
\usepackage[utf8]{inputenc}
\usepackage[russian]{babel}
\usepackage[english=british]{csquotes}


% Пакеты Американского математического общества 
\usepackage{amsmath, amsfonts, amssymb, amsthm, mathtools}




\setmainfont{Times New Roman}

\title{\textbf{\textsc{Компьютерные сети}}}
\author{Боднар Олег Леонидович}
\date{} % Для того чтобы не было даты
\linespread{1.3} % Отступ сверху для титульника

\hypersetup{hidelinks} % Чёрные ссылки
% Цветные ссылки
%\hypersetup{
	%colorlinks=true,
	%linkcolor=blue,
	%filecolor=magenta,
	%urlcolor=cyan,
	%citecolor=green,
	%pdfpagemode=FullScreen,
	%}

\pagestyle{fancy} % Устанавливаем стиль страницы
\fancyhf{}
\lhead{\small\hyperref[sec:toc]{\leftmark}} % Тема слева
\rhead{\thepage} % Страница справа
\renewcommand{\headrulewidth}{1pt} % Убрать линию в контитулах

\begin{document}
	
	\maketitle % Титульник
	\thispagestyle{empty} % Отсутствие номера страницы на титульной странице
	\clearpage % Делаем пустой титульник чтобы перенести оглавление
	\setcounter{page}{1} % Сброс счетчика страниц
	\thispagestyle{empty}
	\tableofcontents\label{sec:toc}
	\clearpage
	
	\titleformat{\section}
	{\normalfont\Large\bfseries}{\thesection}{1em}{}
	
	\titleformat{\subsection}
	{\normalfont\Large\bfseries}{\thesubsection}{1em}{}
	
	\titleformat{\subsubsection}
	{\normalfont\Large\bfseries}{\thesubsubsection}{1em}{}
	
	
	\section{Введение}

    На слияние компьютеров и коммуникационных систем существенно влияет принцип организации компьютерных систем. Некогда доминирующее понятие "вычислительного центра" как комнаты с большим компьютером, к которому пользователи приносят свою работу для обработки, является теперь полностью устаревшим. Старая модель единственного компьютера, служащего всем вычислительным потребностям организации, была заменена на схему, при которой задание выполняет большое количество отдельных, но связанных компьютеров. Эти системы называются \textit{компьютерными сетями}.
    
    \par Термин \textit{компьютерные сети} означает набор автономных компьютеров, связанных одной технологией. Два компьютера называют связанными, если они в состоянии обмениваться информацией.  Соединение не обязательно должно представлять собой медный провод; может использоваться волоконная оптика, микроволны, инфракрасный диапазон и спутников связи. Сети бывают различных размеров, формы и конфигураций. Они обычно соединяются вместе, чтобы создать большие сети, самым известным примером сети сетей является Интернет. 
    
    \par В литературе существует путаница между понятиями компьютерная сеть и \textit{распределённая система}. Основные их различие заключается в том, что в распределённой системе наличие многочисленных автономных компьютеров незаметно для пользователя. С его точки зрения это единая связанная система. Обычно имеется набор программного обеспечения на определённом уровне (над операционной системой), которое называется связующим ПО и отвечает за реализацию этой идеи. Хорошо известный пример этой системы - это Всемирная паутина (World Wide Web), в которой с точки зрения пользователя, всё выглядит как документ (веб-страница). В распределённой системе компьютеры работают вместе, скрывая физические границы между ними и представляясь как единая система между пользователями 
	
	\par В компьютерных сетях нет никакой единой модели, нет никакого программного обеспечения для её реализации. Пользователи имеют дело с реальными машинами, и со стороны вычислительной системы не осуществляется никаких попыток связать их воедино. Скажем, если компьютеры имеют разное аппаратное и программное обеспечение, пользователь не может этого не заметить. Если он хочет запустить программу на удалённой машине, ему придётся явно зарегистрироваться на ней и явно дать задание на запуск. 
	
	\subsection{Применение компьютерных сетей}
	
    Сначала начнём наше обсуждение с таких традиционных вещей, как сети в организациях, затем перейдём к домашним сетям и новым технологиям, связанных с мобильной связью и мобильными пользователями, и закончим социальными вопросами.
    
    \subsubsection{Сети в организациях}
    
    Первоначально компьютеры  работали в изоляции от других, но в некоторый момент управление решило соединить их, чтобы быть в состоянии передавать информацию по всей компании. 
    
    \par Если посмотреть на эту проблему с более общих позиций, то вопросом здесь является совместное использование ресурсов, а целью предоставление доступа к программам, оборудованию и особенно данным для любого пользователя сети, независимо от физического расположения ресурса и пользователя. В качестве примера можно привести сетевой принтер, то есть устройство, доступ к которому может осуществляться с любой рабочей станции сети.
    
    \par Но более важной проблемой, нежели совместное использование физических ресурсов, таких как принтеры и устройства резервного копирования, является совместное использование информации.
    
    \par В маленьких компаниях все компьютеры обычно собраны в пределах одного офиса или, в крайнем случае, одного здания. Если же речь идет о больших фирмах, то и вычислительная техника, и служащие могут быть разбросаны по десяткам представительств в разных странах. Несмотря на это, продавец, находящийся в Нью-Йорке, может запросить и сразу же получить информацию о товарах, имеющихся на складе в Сингапуре. Для соединения сетей, расположенных в разных местах, могут быть использованы сети, называемые VPN (Virtual Private Networks - виртуальные частные сети). Другими словами, тот факт, что пользователь удален от физического хранилища данных на 15 тысяч километров, никак не ограничивает его возможности доступа к этим данным. Можно сказать, что одной из целей сетей является борьба с «тиранией географии».
	
	\begin{figure}[h]
		\centering
		\includegraphics[height=8cm]{img/1.1} 
		\captionsetup{font=footnotesize}
		\caption{Сеть состоящая из двух клиентов и одного сервера} 
	\end{figure}
	
	\par  Проще всего информационную систему компании можно представить себе как совокупность одной или более баз данных с информацией компании и некоторого количества работников, которым удалённо предоставляется информация. В этом случае 
	данные хранятся на мощном компьютере, называемом \textbf{сервером}. Довольно часто сервер располагается в отдельном помещении и обслуживается системным администратором. С другой стороны, компьютеры служащих могут быть менее мощными, они  идентифицируются в сети как \textbf{клиенты}, могут в большом количестве располагаться даже в пределах одного офиса и иметь удалённый доступ к информации и программам, хранящимся на сервере.  Клиентская и серверные машины объединены в сеть, как показано на рис. 1.
    
    \par Такая система называется \textbf{клиент-серверной моделью}. Она используется очень широко и зачастую является основой построения всей сети. Самая популярная реализация - веб-приложение, в котором сервер находится в одном здании и принадлежит одной компании, а также когда они расположены далеко друг от друга. Скажем, когда пользователь получает доступ к интернет-сайту, работает та же модель. При этом веб-сервер играет роль серверной машины, а пользовательский компьютер - клиентской. 
    
    \par Если рассмотреть модель «клиент-сервер» подробнее, то в ней можно выделить два процесса: серверный и клиентский. Обмен информации чаще всего происходит так. Клиент посылает запрос серверу через сеть и начинает ожидать ответ. При принятии запроса сервер выполняет определённые действия или ищет запрашиваемые данные, затем отсылает ответ. Всё это показано на рис. 2. 
    
    \begin{figure}[h]
    	\centering
    	\includegraphics[height=4cm]{img/1.2} 
    	\captionsetup{font=footnotesize}
    	\caption{В модели «клиент-сервер» различают запросы и ответы} 
    \end{figure}
	
	\par Вторая цель работы компьютерной сети связана в большой степени с людьми, чем с информацией или вычислительными машинами. Сеть - \textbf{коммуникационная среда} для работников предприятия. В любой компании почти каждый компьютер умеет принимать и отправлять \textbf{электронную почту (e-mail)}. 
	
	\par Телефонные звонки между служащими могут передаваться по компьютерной сети, вместо телефонной. Эту технологию называют IP-телефонией или VoIP (Voice over IP). Микрофон и динамик в каждом конце могут принадлежать VoIP-включенному телефону или компьютеру сотрудника.  К аудио можно также добавить видео. Совместный доступ к рабочему столу позволяет удалённым сотрудникам видеть и взаимодействовать с графическим монитором. 
	
	\par Третья цель для многих компаний - заниматься коммерцией с помощью электроники, особенно с клиентами и поставщиками. Эту модель называют электронной коммерцией (e-commerce), и она последние годы быстро растёт. Многие компании начали обеспечивать каталог своих товаров и услуг онлайн и делают заказы онлайн. 
	
	\subsubsection{Использование сетей частными лицами}
	
	Немного истории. В 1977 году Кен Олсен (Ken Olsen) был президентом корпорации DEC(Digital Equipment Corporation),  которая на тот момент была второй по величине (после IBM) компанией, производящей компьютерную технику. Когда у него спросили, почему DEC не поддерживает идею создания персональных компьютеров, он сказал: «Я не вижу никакого смысла в том, чтобы в каждом доме стоял компьютер». Возможно, он и был прав, но исторический факт заключается в том, что все оказалось как раз наоборот, а корпорация DEC вообще прекратила свое существование. Зачем люди начали устанавливать компьютеры у себя дома? Изначально основной целью было редактирование текстов и электронные игры. Недавно самой большой причиной купить домашний компьютер был доступ к Интернету. Теперь многие бытовые электронные устройства, такие как цифровые приемники, игровые приставки и радиоприемники с часами, производятся со встроенными компьютерами и компьютерными сетями, особенно беспроводными сетями, и домашние сети широко используются для развлечения, включая слушание, просмотр, и создание музыки, фотографий и видео. 
	
	\par Доступ к Интернету предоставляет домашним пользователям много услуг, от общения с другими людьми до покупки продуктов и услуг. Основная выгода теперь возникает из соединения за пределами дома. Боб Меткэйфл, изобретатель Ethernet, выдвинул гипотезу, что значение сети пропорционально квадрату числа пользователей, потому что это - примерное число различных соединений, которые могут быть сделаны (Гилдер, 1993). Эта гипотеза известна как «закон Меткэйлфа». Она помогает объяснить, как огромная популярность Интернета возникает из его размера.
	
	\begin{figure}[h]
		\centering
		\includegraphics[height=6cm]{img/1.3} 
		\captionsetup{font=footnotesize}
		\caption{В равноранговой сети нет четко определенных клиентов и серверов} 
	\end{figure}
	
	\par Сейчас в интернете можно найти, что угодно на интересующие нас темы. Доступ к большей части информации осуществляется по модели клиент-сервер, но есть и другой популярный тип сетевого обращения, основанный на технологии \textbf{равно-ранговых сетей (peer-to-peer)}, (Parameswaran и др., 2001). Каждый может связаться с каждым, разделение на клиентские и серверные машины в этом случае отсутствует. Это показано на рис. 3. 
	
	\par Многие одноранговые сети, такие как BitTorrent (Cohen, 2003), не имеют никакой центральной базы данных контента. Вместо этого каждый пользователь поддерживает свою собственную базу данных в  местном масштабе и обеспечивает список других людей по соседству, которые являются членами системы. Новый пользователь может пойти к любому существующему участнику, увидеть то, что он имеет, и получить имена других участников, таким образом получая доступ к большему количеству контента и количеству имён. 
	
	\par Коммуникация соединения равноправных узлов ЛВС (локальная вычислительная сеть) часто  задействуется, чтобы использовать музыку и видео. Этого рода коммуникации стали очень популярны примерно в 2000 году, они были реализованы с помощью службы Napster. 
	
	\par В конечном счёте сети очень важны для коммуникации между людьми и создания контента. 
	
    \par 


	
\end{document}
	
