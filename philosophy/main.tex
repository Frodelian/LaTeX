\documentclass[
a4paper, 
fontsize=10pt, 
twoside=false, 

]{kaobook}
\usepackage[T2A]{fontenc}
\usepackage[utf8]{inputenc}
\usepackage[russian]{babel}
\usepackage[english=british]{csquotes}


% Пакеты Американского математического общества 
\usepackage{amsmath, amsfonts, amssymb, amsthm, mathtools}



\setmainfont{Times New Roman}
\title{\textbf{\textsc{История западной философии}}}
\author{Боднар Олег Леонидович}
\date{} % Для того чтобы не было даты
\linespread{1.3} % Отступ сверху для титульного листа

\begin{document}
	
\maketitle % Титульный лист
	\thispagestyle{empty} % Отсутствие номера страницы на титульной странице
	\clearpage % Делаем пустой титульный лист чтобы перенести оглавление
	\setcounter{page}{1} % Сброс счетчика страниц
		
	\tableofcontents
	\clearpage
	\section{Введение}
	\subsection{Предмет философии}
	
	\par В переводе с древнегреческого философия - "любовь и мудрость". Впервые это слово 
	по свидетельству Диогена Лаэртского и Цицерона, употребил Пифагор. Один из первых 
	древнегреческих философов: мудрецов, по мнению Пифагора может быть только Бог, человек 
	может испытывать влечение к мудрости. Поверить справедливость этих свидетельств мы не 
	можем, поскольку Пифагор не оставил после себя никаких записей.  Самое первое 
	"авторизованное" упоминание слова "философии" встречается у Геродота, который приводит 
	письмо Креза к Солону: "...ты, стремясь к мудрости и желая повидать свет, объездил 
	много стран". Древнегреческий философ Платон считал, что впервые придал истинный 
	смысл слову "философия" Сократ. 
	\clearpage
	
	\section{Античная философия}
	
	\subsection{Возникновение философии}
    \subsubsection{Религии древней Греции}	
    Древние греки, или те народы, которые впоследствии стали древними греками, не сомневались
	что душа существует отдельно. Под душой понимали, конечно же, не то, что мы сейчас 
	понимаем под этим словом. Греческое слово ψυχή, «псюхе», иногда возводят к слову 
	ψῦχος, «псюхос», — прохлада, т. е. та прохлада, которая производится посредством нашего 
	дыхания. Эту этимологию будет использовать для своих целей христианский богослов Ориген, 
	утверждавший, что наши души охладели в своей любви к Богу. (Вспомним, что в русском 
	языке слова «душа», «дух», «дышать», «воздух» также имеют общее происхождение.) Греки 
	пытались умилостивить души умерших, устраивали в честь них праздники, из которых 
	впоследствии возникла греческая драма. Ведь если душа принадлежала человеку, который 
	умер насильственной смертью, то она мстила людям (такие души назывались эриниями, или, 
	в римской мифологии, фуриями). Эринии охраняли ворота в Аид, поскольку они никем не 
	могли быть подкуплены. 
    
    \par Особенности греческой религии состояла в том, что под богами греки понимали сущность 
	вещи или явления, в отличии от римской мифологии, где богом было само явление. Скажем, 
	бог моря Посейдон символизировал собой сущность морской стихии, в то время как бог Нептун 
	был само море со всеми его явлениями. Может быть, в этом мы увидим ключ к разгадке 
	феномена греческой философии и поймем, почему философия возникает именно в Древней 
	Греции, а в Древнем Риме философия всегда существовала лишь в форме чисто эклектического 
	восприятия идей греческих философов.
    
    \par Греческая религия не была единым цельным явлением, в ней существовало несколько 
	религий. Среди большого многообразия греческих религий полезно ознакомиться с тремя 
	формами — религией Зевса, религией Деметры и религией Диониса. Проследим, каким образом 
	из этих религий возникают различные направления греческой философии.
    
    \subsubsection{Религия Зевса-Аполлона}
    
    \par Религия Зевса, пожалуй, лучше всего известна, хотя бы потому что основные мифы 
	и положения этой религии изложены в книгах Гомера и Гесиода. Гомера Геродот даже 
	называет создателем греческой религии. Не будем спорить с Геродотом, но, скорее всего, 
	он преувеличивал значение Гомера. У Гомера мы не встречаем систематизированной мифологии 
	или тем более философии. Мифы и некоторые концепции, которые можно назвать философскими, 
	встроены в повествование его «Одиссеи» и «Илиады». Только лишь внимательное чтение 
	позволяет выделить некоторые предфилософские элементы и определить, каково же было 
	мировоззрение самого Гомера.
    
    \par Быть может, самым важным вкладом Гомера в философию (на это обращает внимание еще 
	Аристотель) является постановка им вопроса о первоначале. Он спрашивает: что же было 
	прародителем всего? И отвечает: «Океан всему прародитель» (Илиада, 241). (Океан — 
	это река, которая со всех сторон омывала Землю.) Кроме того, Гомер предлагает и 
	некоторую космологию, утверждая, что существуют три части Вселенной: небо, земля и 
	преисподняя, которая в свою очередь состоит из Аида и Тартара. По Гомеру, земля отстоит 
	от неба на таком же расстоянии, как Тартар отстоит от земли. Венчает все Эфир.
    
    \par Далее, в мифологии Гомера мы можем увидеть и предфилософский анализ явлений. 
	В частности, боги, которые фигурируют в его «Одиссее» и «Илиаде», находятся между 
	собой в родственных связях. И это, конечно же, неслучайно. Напри-мер, бог смерти 
	Танатос является братом бога сна Гипноса: Гомер и его современники, видимо, пытались 
	найти связь между сном и смертью и выражали ее на языке мифологическом, на языке 
	родственной связи между богами.
    
    \par Есть у Гомера и своеобразная антропология, учение о человеке. В человеке 
	Гомер различает две части: душу и тело. Причем душа понимается трояко: душа как 
	«псюхе» — бесплотный образ тела, как бы его копия, только нематериальная, не имеющая 
	плоти, хотя и телесная; душа как «тюмос» — волевое начало в человеке; и душа как «ноос» 
	(в более позднем языке — «нус»), т. е. как ум. Все три вида души существуют только 
	у богов и человека, животные обладают первым и вторым видами души. Однако наиболее
	истинное бытие человека — это его существование на земле. Загробная жизнь хуже земной. 
	В Аиде, пишет Гомер, «только тени умерших людей, сознанья лишенные, реют» (Одиссея, 
	475–476). 
    
    \par Еще один вклад Гомера в философию состоит в том, что боги у него не всесильны. 
	Они подчиняются судьбе, или мойре. Нельзя сказать, что это некий бог судьбы, некая 
	безличная судьба, как бы прообраз понятия закона.
    
    \par Более разработанная концепция — и философская и космологическая — содержится в 
	работах Гесиода, младшего современника Гомера. Перу Гесиода принадлежат два дошедших 
	до нас произведения — «Труды и дни» и «Теогония». «Труды и дни» посвящены истории 
	развития человечества, описанию прошедшего золотого века и того упадка, который 
	переживало человечество во времена Гесиода. В «Теогонии» же Гесиод показывает развернутую 
	картину возникновения богов. И так же как Гомер, он ставит вопрос о начале — уже не 
	просто о субстанциальном начале, но и о начале хронологическом. Гесиода волнует вопрос: 
	что было в самом начале, лежит в основе мира и явилось его порождающей причиной? 
	Ответ его вы-глядит следующим образом: 
    \\
    
    \textit{Прежде всего во вселенной Хаос зародился, а следом }
    \par \textit{Широкогрудая Гея, всеобщий приют безопасный,} 
    \par \textit{Сумрачный Тартар, в земных залегающий недрах глубоких, }
    \par \textit{И, между вечными всеми богами прекраснейший,- Эрос. }
    \par \textit{ (Теогония, 116)}
     
    
	\section{Основная часть}
	Тут будет основная часть текста.
	\clearpage
	\subsection{Подраздел}
	Тут будет текст подраздела.
	\clearpage
	\section{Заключение}
	Здесь будет заключение.
	
\end{document}