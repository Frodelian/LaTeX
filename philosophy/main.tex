\documentclass[
a4paper, 
fontsize=10pt, 
twoside=false, 

]{kaobook}
\usepackage[T2A]{fontenc}
\usepackage[utf8]{inputenc}
\usepackage[russian]{babel}
\usepackage[english=british]{csquotes}


% Пакеты Американского математического общества 
\usepackage{amsmath, amsfonts, amssymb, amsthm, mathtools}



\setmainfont{Times New Roman}
\title{\textbf{\textsc{История западной философии}}}
\author{Боднар Олег Леонидович}
\date{} % Для того чтобы не было даты
\linespread{1.3} % Отступ сверху для титульного листа

\begin{document}
	
    \maketitle % Титульный лист
	\thispagestyle{empty} % Отсутствие номера страницы на титульной странице
	\clearpage % Делаем пустой титульный лист чтобы перенести оглавление
	\setcounter{page}{1} % Сброс счетчика страниц
		
	\tableofcontents
	\clearpage
	\section{Введение}
	\subsection{Предмет философии}
	
	\par В переводе с древнегреческого философия - "любовь и мудрость". Впервые это слово 
	по свидетельству Диогена Лаэртского и Цицерона, употребил Пифагор. Один из первых 
	древнегреческих философов: мудрецов, по мнению Пифагора может быть только Бог, человек 
	может испытывать влечение к мудрости. Поверить справедливость этих свидетельств мы не 
	можем, поскольку Пифагор не оставил после себя никаких записей.  Самое первое 
	"авторизованное" упоминание слова "философии" встречается у Геродота, который приводит 
	письмо Креза к Солону: "...ты, стремясь к мудрости и желая повидать свет, объездил 
	много стран". Древнегреческий философ Платон считал, что впервые придал истинный 
	смысл слову "философия" Сократ. 
	\clearpage
	
	\section{Античная философия}
	
	\subsection{Возникновение философии}
    \subsubsection{Религии древней Греции}	
    
	Древние греки, не сомневались что душа существует отдельно. Под душой понимали, 
	конечно же, не то, что мы сейчас 
	понимаем под этим словом. \textbf{Греческое слово ψυχή, «псюхе», иногда возводят к слову 
	ψυχος, «псюхос», — прохлада, т. е. та прохлада, которая производится посредством нашего 
	дыхания.} Эту этимологию будет использовать для своих целей христианский богослов Ориген, 
	утверждавший, что наши души охладели в своей любви к Богу. (Вспомним, что в русском 
	языке слова «душа», «дух», «дышать», «воздух» также имеют общее происхождение.) Греки 
	пытались умилостивить души умерших, устраивали в честь них праздники, из которых 
	впоследствии возникла греческая драма. Ведь если душа принадлежала человеку, который 
	умер насильственной смертью, то она мстила людям (такие души назывались эриниями, или, 
	в римской мифологии, фуриями). Эринии охраняли ворота в Аид, поскольку они никем не 
	могли быть подкуплены. 
    
    \par \textbf{Особенность греческой религии состояла в том, что под богами греки понимали 
	сущность вещи или явления, в отличии от римской мифологии, где богом было само явление.} 
	Скажем, 
	бог моря Посейдон символизировал собой сущность морской стихии, в то время как бог Нептун 
	был само море со всеми его явлениями. Может быть, в этом мы увидим ключ к разгадке 
	феномена греческой философии и поймем, почему философия возникает именно в Древней 
	Греции, а в Древнем Риме философия всегда существовала лишь в форме чисто эклектического 
	восприятия идей греческих философов.
    
    \par Греческая религия не была единым цельным явлением, в ней существовало несколько 
	религий. Среди большого многообразия греческих религий полезно ознакомиться с тремя 
	формами — религией Зевса, религией Деметры и религией Диониса. Проследим, каким образом 
	из этих религий возникают различные направления греческой философии.
    
    \subsubsection{Религия Зевса-Аполлона}
    
    \par \textbf{Религия Зевса, пожалуй, лучше всего известна, хотя бы потому что основные мифы 
	и положения этой религии изложены в книгах Гомера и Гесиода. Гомера Геродот даже 
	называет создателем греческой религии.} Не будем спорить с Геродотом, но, скорее всего, 
	он преувеличивал значение Гомера. У Гомера мы не встречаем систематизированной мифологии 
	или тем более философии. Мифы и некоторые концепции, которые можно назвать философскими, 
	встроены в повествование его «Одиссеи» и «Илиады». Только лишь внимательное чтение 
	позволяет выделить некоторые предфилософские элементы и определить, каково же было 
	мировоззрение самого Гомера.
    
    \par \textbf{Быть может, самым важным вкладом Гомера в философию (на это обращает 
	внимание еще 
	Аристотель) является постановка им вопроса о первоначале. Он спрашивает: что же было 
	прародителем всего? И отвечает: «Океан всему прародитель» (Илиада, 241). (Океан — 
	это река, которая со всех сторон омывала Землю.) Кроме того, Гомер предлагает и 
	некоторую космологию, утверждая, что существуют три части Вселенной: небо, земля и 
	преисподняя}, которая в свою очередь состоит из Аида и Тартара. По Гомеру, земля отстоит 
	от неба на таком же расстоянии, как Тартар отстоит от земли. Венчает все Эфир.
    
    \par \textbf{Далее, в мифологии Гомера мы можем увидеть и предфилософский анализ явлений. 
	В частности, боги, которые фигурируют в его «Одиссее» и «Илиаде», находятся между 
	собой в родственных связях.} И это, конечно же, неслучайно. Напри-мер, бог смерти 
	Танатос является братом бога сна Гипноса: Гомер и его современники, видимо, пытались 
	найти связь между сном и смертью и выражали ее на языке мифологическом, на языке 
	родственной связи между богами.
    
    \par Есть у Гомера и своеобразная антропология, учение о человеке. \textbf{В человеке 
	Гомер различает две части: душу и тело.} Причем душа понимается трояко: душа как 
	«псюхе» — бесплотный образ тела, как бы его копия, только нематериальная, не имеющая 
	плоти, хотя и телесная; душа как «тюмос» — волевое начало в человеке; и душа как «ноос» 
	(в более позднем языке — «нус»), т. е. как ум. Все три вида души существуют только 
	у богов и человека, животные обладают первым и вторым видами души. \textbf{Однако наиболее
	истинное бытие человека — это его существование на земле. Загробная жизнь хуже земной.} 
	В Аиде, пишет Гомер, «только тени умерших людей, сознанья лишенные, реют» (Одиссея, 
	475–476). 
    
    \par \textbf{Еще один вклад Гомера в философию состоит в том, что боги у него не всесильны. 
	Они подчиняются судьбе, или мойре. Нельзя сказать, что это некий бог судьбы, некая 
	безличная судьба, как бы прообраз понятия закона.}
    
    \par \textbf{Более разработанная концепция — и философская и космологическая — содержится в 
	работах Гесиода, младшего современника Гомера.} Перу Гесиода принадлежат два дошедших 
	до нас произведения — «Труды и дни» и «Теогония». «Труды и дни» посвящены истории 
	развития человечества, описанию прошедшего золотого века и того упадка, который 
	переживало человечество во времена Гесиода. В «Теогонии» же Гесиод показывает развернутую 
	картину возникновения богов. И так же как Гомер, он ставит вопрос о начале — уже не 
	просто о субстанциальном начале, но и о начале хронологическом. \textbf{Гесиода волнует вопрос: 
	что было в самом начале, лежит в основе мира и явилось его порождающей причиной?}
	Ответ его вы-глядит следующим образом: 
    \\
    
    \textit{Прежде всего во вселенной Хаос зародился, а следом }
    \par \textit{Широкогрудая Гея, всеобщий приют безопасный,} 
    \par \textit{Сумрачный Тартар, в земных залегающий недрах глубоких, }
    \par \textit{И, между вечными всеми богами прекраснейший,- Эрос. }
    \par \textit{ (Теогония, 116)}
     
    \par \textbf{Таким образом порождающей причиной У Гесиода является хаос, который следует понимать
	не как некий беспорядок, а как бездну. Точнее, «хаос» — это некая пропасть между землей
    и небом. Впоследствии из хаоса рождаются боги - } Гея (земля), Тартар, Эрос, Нюкта 
	(ночь) и Эреб (мрак). Гея порождает из себя Ура-на, т. е. небо, нимф и Понт (море). 
    В дальнейшем Гея и
	Уран рождают титанов, киклопов и гекатонхейров (сторуких). Уран стыдится своих
    отнюдь не прекрасных детей и не выпускает их из чрева матери Геи. Гея страдает, 
	ненавидит Урана и тайком от него рож-дает одного титана — Крона. Одновременно 
	с этим появляются такие боги, как Старость, Смерть, Печаль и т. д. Крон оскопляет 
	Урана и выпускает всех остальных титанов из чрева матери-земли.
	На следующем этапе Крон и титанида Рея рождают известных нам по гомеровским
	мифам богов-олимпийцев. Однако Крон, помня, что он сделал со своим отцом, 
	подозревает, что и его дети сделают с ним то же самое, и пожирает своих детей.
	Рея вместо одного своего сына подсовывает ему камень, и Зевс оказывается 
	таким образом уцелевшим. Зевсу освобожденные им гекатонхейры дают свое оружие — 
	гром и молнию, и при помощи грома и молнии Зевс ниспровергает титанов и 
	становится верховным богом греческого пантеона. Он сбрасывает в Тартар всех 
	титанов и в качестве их тюремщиков — гекатонхейров. 
	Таким образом, Гесиод рассказал о том, что произошло до тех событий, которые 
	описываются у Гомера. \textbf{Гесиод гораздо в большей степени, чем Гомер, 
	систематизирует историю возникновения мира, прослеживая ее в виде происхождения богов.}
	В дальнейшем у Зевса также рождаются дети, и один из его сыновей — Аполлон — 
	становится другим верховным богом греческого пантеона. Религия Зевса и Аполлона 
	стала практически официальной религией Древней Греции. Известен храм Аполлона в 
	Дельфах, где прорицательницы-пифии вещали, сидя на треножнике, волю богов, и в 
	первую очередь Аполлона.

	\subsubsection{Религия Демитры}

	\par Эта религия вырастает из мифа, согласно которому у Деметры Аид, брат Зевса,
	похищает ее дочь Кору, или Персефону. Деметра обращается к другим богам, 
	и те, не желая портить отношения с Аидом, приходят к компромиссному решению, 
	согласно которому Персефона должна попеременно жить то на земле, то в Аиде. 
	\textbf{Таким образом, у людей появляется посредник между Аидом и землей, 
	между загробным миром и земной жизнью, и люди узнают о том, что 
	ожидает их после смерти.} Вследствие этого возникают таинства Деметры и Коры. 
	Эта религия была эзотеричной, в эти таинства-мистерии посвящались не все. 
	В частности, хорошо известны Елевсинские мистерии. 
	Посвященные в религию Деметры достигали посмертного существования. 
	Следы этой религии можно найти в трагедиях Эсхила.
	
	\subsubsection{Религия Диониса. Орфики}

	\par Религия Диониса тесно связана с религией Деметры. 
	\textbf{В основе этой религии лежит поклонение богу Дионису, впоследствии ставшему богом вина.
	Он стал богом вина, в частности, потому, что поклонение Дионису проходило в 
	форме употребления вина, неистовых плясок, т. е. того, что стало называться 
	вакханалиями по другому имени Диониса — Вакх.}
	Служительницами бога Вакха, или Диониса, были вакханки. 
	\textbf{Во время вакханалий люди, участвовавшие в них, оказывались в состоянии экстаза 
	(термин «экстаз» происходит от греческого «экстасис» — выхождение вне 
	себя, т. е. как бы выхождение из своего собственного тела) и обнаруживали, 
	что кроме тела у человека существует и душа и что душа может существовать 
	независимо от тела. А поскольку экстаз сопровождался различного рода 
	приятными ощущениями и оставлял после себя ощущение истинности, то сторонники 
	этой религии, дионисийцы, приходили к выводу, что состояние это является гораздо 
	более истинным и достойным существования, чем состояние в теле. Появляется концепция, 
	согласно которой существование в земном теле объявляется неистинным и тело есть 
	могила души.} Развил и систематизировал религию Диониса Орфей, 
	легендарный греческий герой. Согласно мифу, Орфей потерял свою возлюбленную 
	Эвридику, умершую от укуса змеи, и, взяв с собой свою лиру, отправился к Аиду. 
	Игрой на лире он усыпил стоглавого пса Цербера, умилостивил неумолимых 
	эриний и уговорил Персефону, которая уже давно живет у Аида и является 
	владычицей подземного царства, отпустить Эвридику. 
	Персефона согласилась с одним условием, чтобы Орфей 
	шел впереди Эвридики и не оборачивался на свою возлюбленную. 
	Орфей уже почти вышел, но не выдержал и обернулся, так что Эвридика 
	осталась в царстве Аида, а Орфей вышел один. 
	Орфей так и остался однолюбом, верным Эвридике, и во время одной из 
	вакханалий был растерзан вакханками.
    
	\par \textbf{Характерно отличие религий Зевса и Диониса: у Гомера мы читаем, что 
	земная жизнь лучше и ценнее жизни загробной, что в Аиде обитают не души, а лишь 
	тени; Орфей же утверждает обратное, что тело есть могила души и загробная жизнь 
	является истинной и лучшей долей для человека.}
    
	\par \textbf{Целью жизни орфики считали освобождение души от тела, 
	разделяя при этом точку зрения, что душа после смерти вновь воплощается в тело
	 — человека или животного, — согласно той жизни, которую человек вел до своей смерти.}
	Посвящение в таинства Диониса служило цели избавления человека от бесконечного 
	возвращения в тело, достижения вечной блаженной жизни в царстве мертвых. 
    
	\par Из этих двух религий — дионисийской и аполлоновской 
	— возникают две начальные школы древнегреческой философии — италийская, 
	у истоков которой стоял Пифагор, и милетская, первым представителем которой является 
	Фалес. 
	
	\subsection{Семь мудрецов}

	\par

	

	\section{Основная часть}
	Тут будет основная часть текста.
	\clearpage
	\subsection{Подраздел}
	Тут будет текст подраздела.
	\clearpage
	\section{Заключение}
	Здесь будет заключение.
	
\end{document}