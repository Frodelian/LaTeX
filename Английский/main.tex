\documentclass[
a4paper, 
fontsize=10pt, 
twoside=false, 

]{kaobook}
\usepackage[T2A]{fontenc}
\usepackage[utf8]{inputenc}
\usepackage[russian]{babel}
\usepackage[english=british]{csquotes}


% Пакеты Американского математического общества 
\usepackage{amsmath, amsfonts, amssymb, amsthm, mathtools}




\setmainfont{Times New Roman}

\title{\textbf{\textsc{Английский}}}
\author{}
\date{} % Для того чтобы не было даты
\linespread{1.3} % Отступ сверху для титульника

\hypersetup{hidelinks} % Чёрные ссылки
% Цветные ссылки
%\hypersetup{
	%colorlinks=true,
	%linkcolor=blue,
	%filecolor=magenta,
	%urlcolor=cyan,
	%citecolor=green,
	%pdfpagemode=FullScreen,
	%}

\pagestyle{fancy} % Устанавливаем стиль страницы
\fancyhf{}
\lhead{\small\hyperref[sec:toc]{\leftmark}} % Тема слева
\rhead{\thepage} % Страница справа
\renewcommand{\headrulewidth}{1pt} % Убрать линию в контитулах

\begin{document}
	
	\maketitle % Титульник
	\thispagestyle{empty} % Отсутствие номера страницы на титульной странице
	\clearpage % Делаем пустой титульник чтобы перенести оглавление
	\setcounter{page}{1} % Сброс счетчика страниц
	\thispagestyle{empty}
	
	\renewcommand{\contentsname}{Содержание} % Изменяем название оглавления
	\tableofcontents\label{sec:toc}
	\clearpage
	\section{Задания}
	\subsection{Задание 1}
	\par 1) A biological virus is a very small, simple organism that infects living cells, known as the host, by \textbf{attaching} itself to them and using them to \textbf{reproduce} itself. This often \textbf{causes harm} to the host cells.
	\\
	Перевод: Биологический вирус — это очень маленький, простой организм, который заражает живые клетки, известные как хозяева, \textit{прикрепляясь} к ним и используя их для \textit{воспроизводства} себя. Это часто \textit{наносит вред} клеткам хозяина.
	
	
	\par 2) The virus may also contain a \textbf{payload} that remains dormant until a trigger event activates it, such as the user pressing a particular key.
	\\
	Перевод: Вирус также может содержать \textit{загрузку}, которая остается неактивной до тех пор, пока какое-либо триггерное событие не активирует ее, например, нажатие пользователем определенной клавиши.
	
	\par 3) To be a virus, a program only needs to have a \textbf{reproduction routine} that enables it to infect other programs.
	\\
	Перевод: Чтобы быть вирусом, программе нужно иметь \textit{рутину воспроизводства}, которая позволяет ей заражать другие программы.
	
	\par 4) For the system to work, two parties \textbf{engaging} in a secure transaction must know each other's \textbf{public} keys. \textbf{Private} keys, however, are closely guarded secrets known only to their owners.
	\\
	Перевод: Для работы системы две стороны, \textit{участвующие} в безопасной транзакции, должны знать друг друга \textit{открытые} ключи. \textit{Закрытые} ключи, однако, являются строго охраняемыми секретами, известными только их владельцам.
	
	\par 5) The dynamics of the Web dictate that a user \textbf{authentication} system must exist. This can be done using digital \textbf{certificates}.
	\\
	Перевод: Динамика Интернета диктует, что должна существовать система \textit{аутентификации} пользователей. Это может быть сделано с использованием цифровых \textit{сертификатов}.
	
	\par 6) To make the right decisions we must be able to \textbf{handle} and administer information safely and securely. It also means we must have systems that are \textbf{available} whenever we need them, and that will produce information we can \textbf{rely} on.
	\\
	Перевод: Чтобы принимать правильные решения, мы должны уметь \textit{обрабатывать} и обеспечивать безопасность информации. Это также означает, что у нас должны быть системы, которые будут \textit{доступны} всякий раз, когда нам это нужно, и которые будут производить информацию, на которую мы можем \textit{полагаться}.
	
	\par 7) A large percentage of computer data loss is \textbf{attributed} to accidental error. Accountability for use of assets (systems, data) is a strong \textbf{deterrent} to any wrongdoing.
	\\
	Перевод: Большой процент потери компьютерных данных \textit{приписывается} случайным ошибкам. Ответственность за использование активов (систем, данных) является сильным \textit{сдерживающим} фактором для любого правонарушения.
	
	\par 8) A successful security program consists of a number of \textbf{interrelated key elements}, each of which has a definite purpose and is supported by management.
	\\
	Перевод: Успешная программа безопасности состоит из множества \textit{взаимосвязанных ключевых элементов}, каждый из которых имеет определенную цель и поддерживается руководством.
	
	\par 9) A \textbf{disgruntled} employee is one who works for or used to work for an organization and wants to cause harm or embarrassment to the organization itself.
	\\
	Перевод: \textit{Недовольный} сотрудник — это тот, кто работает или работал в организации и хочет причинить вред или смущение самой организации.
	
	\par 10) The opportunity for a dishonest or disgruntled employee to \textbf{exploit} the organization increases as the network grows.
	\\
	Перевод: Возможность для недобросовестного или недовольного сотрудника \textit{эксплуатировать} организацию возрастает по мере роста сети.
	
	\par 11) Physical protection of network components from theft and from \textbf{hostile environment} must be considered.
	\\
	Перевод: Физическая защита компонентов сети от кражи и от \textit{враждебной среды} должна быть учтена.
	
	\par 12) The biggest \textbf{exposure} of information systems is due to errors caused by honest employees who make mistakes in data entry, data update changes to applications, etc.
	\\
	Перевод: Наибольшая \textit{подверженность} информационных систем возникает из-за ошибок, допущенных добросовестными сотрудниками, которые делают ошибки при вводе данных, изменениях обновление данных в приложениях и т. д.
	\subsection{Задание 2}
	
	
	
	
	\subsection{Словарь}
	
	\par Handle — обрабатывать; управлять (применяется в контексте управления информацией или системами).
	\par Exploit — эксплуатировать; использовать (часто используется в контексте уязвимостей или ресурсов).
	\par Public — открытый; публичный (относится к информации или ключам, доступным для всех).
	\par Authentication — аутентификация; удостоверение личности (процесс проверки личности пользователя).
	\par Disgruntled — недовольный (человек, работающий в организации и имеющий негативные чувства, часто в контексте потенциальной угрозы).
	\par Exposure — подверженность; открытость (в контексте рисков и уязвимостей).
	\par Payload — загрузка; полезная нагрузка (вирусная программа, которая выполняет задачи после заражения).
	\par Interrelated key elements — взаимосвязанные ключевые элементы (части системы безопасности, которые работают вместе).
	\par Attributed — приписываемый; относимый (часто используется в контексте причин и следствий).
	\par Available — доступный (информация или ресурсы, которые могут быть использованы).
	\par Hostile environment — враждебная среда (условия или ситуации, представляющие угрозу безопасности).
	\par Private — частный (информация, доступная только определенным лицам или защищенная от общего использования).
	\par Attaching — прикрепление (процесс соединения вируса с программой).
	\par Rely — полагаться (на что-то; например, на безопасность систем или информации).
	\par Reproduce — воспроизводить (процесс создания копий чего-либо, как в случае вирусов).
	\par Deterrent — сдерживающий фактор (что-то, что предотвращает нежелательные действия, например, преступления).
	\par Reproduction routine — рутина воспроизводства (алгоритм, позволяющий вирусу создавать свои копии).
	\par Causes harm — причиняет вред (описывает действия, которые наносят ущерб).
	\par Engaging — вовлечь; привлечь (участие в чем-то; например, в транзакции).
	\par Certificates — сертификаты (цифровые документы, используемые для аутентификации и безопасности).
	\par Information assets — информационные активы
	\par Encryption — шифрование
	\par Payload — полезная нагрузка
	\par Cost/benefit — затраты/выгода
	\par Infects — заражает
	\par Tenets — принципы / основополагающие положения
	\par Unauthorized — несанкционированный
	\par Public key — открытый ключ
	\par Reproduction routine — рутина воспроизводства
	\par Host — хост / узел
	\par Tampered — подделанный / искаженный
	\par Protecting — защита
	\par Security objectives — цели безопасности
    \par Protective — защитный
    \par Encrypted — зашифрованный
    \par Sequence — последовательность
    \par Message integrity — целостность сообщения
    \par Occurrence — случ occurrence
    \par Destination — пункт назначения
    \par Gibberish — бессмыслица
    \par Executed — выполненный / осуществленный
    \par Reproduce — воспроизводить
    \par Impostor — самозванец
    \par Authentication — аутентификация
    \par
    \par
    \par
    \par
    \par
    \par
    \par
	
\end{document}