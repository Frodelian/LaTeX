\documentclass[
a4paper, 
fontsize=10pt, 
twoside=false, 

]{kaobook}
\usepackage[T2A]{fontenc}
\usepackage[utf8]{inputenc}
\usepackage[russian]{babel}
\usepackage[english=british]{csquotes}


% Пакеты Американского математического общества 
\usepackage{amsmath, amsfonts, amssymb, amsthm, mathtools}




\begin{document}
	
	\section{Некоторые необходимые определения и понятия}

    \subsection{Алфавит и язык}   

    \begin{definition}
    \textbf{Алфавит} $\boldsymbol{A}$ - это конечный или бесконечный набор символов $\boldsymbol{A = \{a_1, a_2, ... , a_n, ...\}}$
    \end{definition}  
    
    \begin{definition}
    \textbf{Слово} - это конечный упорядоченный набор символов алфавита. Выделяется специальное пустое слово \(\Lambda\) (слово, не содержащее символов).
    \\
    Пусть $\boldsymbol{A^*}$ - множество всех слов в алфавите $\boldsymbol{A}$ 
    \end{definition}
    
    \begin{definition}
    \textbf{Язык} $\boldsymbol{L}$ - это подмножество $\boldsymbol{A^*}$. ($\boldsymbol{L}$ \(\subseteq \) $\boldsymbol{A^*}$)   
    \end{definition}
    
    \begin{definition}
    
    \end{definition}  
    
    \subsection{Теория графов}
 
    
    \begin{definition}
   	\textbf{Граф} - это топологическая модель, которая состоит из множества вершин $\boldsymbol{V}$, $\boldsymbol{|V| = n}$ и множества соединяющих их рёбер $\boldsymbol{E}$, $\boldsymbol{|E| = m}$. При этом значение имеет только сам факт, какая вершина с какой соединена. Обозначим простой граф через $\boldsymbol{G = (V, E)}$.
    \end{definition}
    
   \begin{definition}
   	
   \end{definition}
    
    
    \clearpage
    
  

	\section{Задачи и алгоритмы}

	\subsection{Задача}
	 
	\begin{definition}
	\textbf{Задача} - выбор наилучшего множества параметров и связей для достижения некоторой цели.
	\\
	ЗАДАЧА $\rightarrow$ УСЛОВИЕ $\rightarrow$ ПАРАМЕТРЫ $\rightarrow$ ВОПРОСЫ
    \end{definition}
    
    \textbf{Параметры задачи:}
    
    \par -- числовые(непрерывные):
    работа с числами, операциями над числами, уравнениями и неравенствами, а также математическими моделями с числовыми данными.
    \par -- структурные(комбинаторные, дискретные):
    работа с подсчётом, упорядочиванием и комбинированием объектов в дискретном пространстве.\\
     
    \par Если не фиксируем числовые значения, то это задача \textit{"массовая"}. 
    \par Если мы зафиксировали числовые значения, то это \textit{"индивидуальная"} задача. \\

	\par Задача имеет \textit{цель} (т.е. общий вопрос на который требуется дать ответ), из которой следует \textit{форма задачи}.
	
	\par \textit{Цель} - форма задачи. \\
	
	\textbf{Формы задачи:}
    
    \par 1) \textit{Распознавание}:
    Ответ, вариант - "да" или "нет".
    
    \par 2) \textit{Оптимизация}:
    Пусть $\boldsymbol{F}$ - произвольное множество (область допустимых точек) и $\boldsymbol{c(x)}$ - функция стоимости, отображает элементы $\boldsymbol{F}$ на множество действительных чисел. 
    \\
    Задача: требуется найти такую точку $\boldsymbol{x*}$ из $\boldsymbol{F}$, на которой значение функции $\boldsymbol{c(x*)}$ обладает определённым свойством, например максимально, минимально и т.д.
    
    \par 3) \textit{Вычислительная форма}:
    Здесь нужно указывать не сам объект, а его числовое значение. Т.е каждому $\boldsymbol{x}$ сопоставлен $\boldsymbol{c(x)}$, необходимо найти $\boldsymbol{c(x)}$ или $\boldsymbol{c(x*)}$, т.е. найти не сам объект, а его значение.
    
    \par 4) \textit{Перечислительная форма}: 
    Задачи где требуется найти все или определённое количество решений, ответить на вопрос - сколько?\\
    
    \par \textit{Объект} - сущность предназначенная для идентификации объекта, имеющая числовое значение в виде параметра. \textit{Параметр} - сущность принимающая числовые значения, описывает мощность объекта. Они не равноправны: сначала должны быть заданы объекты, а затем уже параметры.
    
    \clearpage

    \textbf{Примеры задач:}
    
    \par 1) \textit{Поиск максимального числа}
    \\
    Вход: заданы $\boldsymbol{N}$ целых чисел $\boldsymbol{a_1, a_2, ... , a_n}$.
    \\
    Вопрос: найти максимальное число(максимальный элемент)?
    \\
    Если всем этим объектам и параметрам присвоены конкретные значения, то получается индивидуальная задача. В ситуации, когда
    речь идет о произвольных значениях объектов, мы имеем дело с массовой задачей. \\
    Форма задачи - оптимизация, если поиск объекта экстремума(параметр $\boldsymbol{a_n}$, объект $\boldsymbol{N}$).\\
    Форма задачи - вычислительная, если надо найти величину параметра(числовое значение).\\
    Форма задачи - распознавания, если требуется ответить на вопрос, существует ли среди объектов $\boldsymbol{a_1, a_2, ... , a_n}$ такой, значение которого не меньше, чем $\boldsymbol{B}$?
    
    \par 2) \textit{Разложение на множители}
    \\
    Вход: задано $\boldsymbol{N}$ чисел. \\
    Вопрос: найти $\boldsymbol{a_1, a_2, ... , a_s}$, $\boldsymbol{p_1, p_2, ... , p_s}$: $\boldsymbol{N = {a_1}^{p_1}, {a_2}^{p_2}, ... , {a_s}^{p_s} }$\\
    Под вычислительной формой в задаче можно понимать нахождение всех простых делителей с их кратностями.
    \par 3) \textit{Задача о простоте числа} \\
    Вход:задано число $\boldsymbol{N}$.\\
    Вопрос: число $\boldsymbol{N}$ простое? \\
    Форма задачи - распознавание.
    \par 4) \textit{Задача о гамильтоновом цикле} \\
    Вход: простой неориентированный граф $\boldsymbol{G = (V, E)}$.\\
    Вопрос: существуют ли в графе гамильтонов цикл?\\
    Форма задачи - распознавание.\\
    Форма задачи - перечислительная, в задаче требуется найти число различных гамильтоновых циклов графа.
    \par 5) \textit{Задача о коммивояжёре} \\
    Вход: граф $\boldsymbol{G = (V, E)}$, матрица весов рёбер $\boldsymbol{A = {(a_{ij})}_{n\times n}}$, $\boldsymbol{|V| = n}$. \\
    Вопрос: найти гамильтонов цикл экстремальной длины(минимальной или максимальной).\\
    Форма задачи - оптимизация. \\
    Форма задачи - вычислительная, вариант задачи состоит, например, в нахождении длины минимального гамильтонова цикла(представление числа).\\ 
    Форма задачи - распознавание, если на входе число $\boldsymbol{B}$ и надо найти число $\boldsymbol{\le B}$.\\
    Форма задачи - перечислительная, вариант состоит в нахождении количества минимальных гамильтоновых циклов.
    \par 6) \textit{Задача о выполнимости КНФ} \\
    Вход: Задана булева функция $\boldsymbol{F}$ от $\boldsymbol{n}$ переменных в виде конъюнктивной нормальной формы $\boldsymbol{K(x_1, x_2, ... , x_n)}$.  \\
    Вопрос: существует ли набор $\boldsymbol{(x_1, x_2, ... , x_n)\in F^n: K(x_1, x_2, ... , x_n) = 1}$?\\
    Форма задачи - распознавание.
    \par 7) \textit{Клика} \\
    Вход: граф $\boldsymbol{G = (V, E)}$ и число $\boldsymbol{B}$. \\
    Вопрос: найти размер максимальной клики? \\
    Форма задачи - вычислительная. \\
    Форма задачи - распознавание, если она отвечает на вопрос, существует ли клика размера $\boldsymbol{\ge B}$? \\
    Форма задачи - оптимизационная, если в ней нужно найти максимальную клику.
    \par 8) \textit{Кратчайший путь} \\
    Вход: граф $\boldsymbol{G = (V, E)}$; набор длин рёбер $\boldsymbol{w_1, w_2, ... , w_n}$;
    $\boldsymbol{|V| = n}$; $\boldsymbol{|E| = m}$; $\boldsymbol{s,t \in V}$. \\
    Вопрос: найти кратчайший путь из из $\boldsymbol{s}$ в $\boldsymbol{t}$. \\
    Форма задачи - оптимизация.
    \par 9) \textit{Кратчайший остов} \\
    Вход: граф $\boldsymbol{G = (V, E)}$; набор длин рёбер $\boldsymbol{w_1, w_2, ... , w_n}$. \\
    Вопрос: Найти кратчайший остов. \\
    Форма задачи - оптимизация.
    \par 10) \textit{Изоморфизм графов} вершины \\
    Вход: графы $\boldsymbol{G = (V, E)}$, $\boldsymbol{H = (U, W)}$; вершины $\boldsymbol{|V| = |U| = n}$, рёбра $\boldsymbol{|E| = |W| = m}$. \\
    Вопрос: $\boldsymbol{G\backsim H}$?\\
    Форма задачи - распознавание.
  
\subsection{Алгоритм}

    \par \textit{Этапы нахождения алгоритма}:
    \par 1) Предложение - в результате опыта или угадывания мы в качестве предлагаемого решения выбираем некоторый объект. В том случае, когда правильное предложение возникает в результате отгадки, интуиции, опыта, вводят понятие оракула. \textit{Оракул} - это абстрактная сущность, которая мгновенно, без затрат ресурсов (памяти, времени и т.п.) способна предоставить предложение. Конечно, его можно проверить, но оракул, на то и оракул, чтобы не ошибаться.
    \par 2) Проверки - убеждаемся, что данный объект является требуемым решением задачи.

\section{Нормальные алгорифмы Маркова}
\clearpage

\section{Машины Тьюринга}
\clearpage

\subsection{Одноленточная машина Тьюринга}
\clearpage

\subsection{Многоленточная машина Тьюринга}
\clearpage

\subsection{Недетерминированная машина Тьюринга}
\clearpage

\subsection{Оракульная машина Тьюринга}
\clearpage

\subsection{Равнодоступная адресная машина}
\clearpage

\section{Сравнение различных формальных схем}
\clearpage

\subsection{Кодировки входных данных}
\clearpage

\subsection{О мерах сложности}
\clearpage

\subsection{Теоремы сравнения}
\clearpage

\subsection{Полиномиальные и неполиномиальные оценки сложности}
\clearpage

\section{Сложность алгоритмов некоторых задач}
\clearpage

\section{Теория NP - полноты}
\clearpage

\subsection{Классы P и NP}
\clearpage

\section{Сводимость задач}
\clearpage

\subsection{Смысл сводимости}
\clearpage

\subsection{Полиномиальная сводимость}
\clearpage

\subsection{Сводимость по Тьюрингу}
\clearpage

\section{Теорема Кука}
\clearpage

\section{Схемы из функциональных элементов}
\clearpage

\subsection{Оценки сложности СФЭ}
\clearpage
	
\end{document}