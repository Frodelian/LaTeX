\documentclass[
a4paper, 
fontsize=10pt, 
twoside=false, 

]{kaobook}
\usepackage[T2A]{fontenc}
\usepackage[utf8]{inputenc}
\usepackage[russian]{babel}
\usepackage[english=british]{csquotes}


% Пакеты Американского математического общества 
\usepackage{amsmath, amsfonts, amssymb, amsthm, mathtools}




\setmainfont{Times New Roman}

\title{\textbf{\textsc{Электроника и схемотехника}}}
\author{Боднар Олег Леонидович}
\date{} % Для того чтобы не было даты
\linespread{1.3} % Отступ сверху для титульника

\hypersetup{hidelinks} % Чёрные ссылки
% Цветные ссылки
%\hypersetup{
    %colorlinks=true,
    %linkcolor=blue,
    %filecolor=magenta,
    %urlcolor=cyan,
    %citecolor=green,
    %pdfpagemode=FullScreen,
%}

\pagestyle{fancy} % Устанавливаем стиль страницы
\fancyhf{}
\lhead{\small\hyperref[sec:toc]{\leftmark}} % Тема слева
\rhead{\thepage} % Страница справа
\renewcommand{\headrulewidth}{1pt} % Убрать линию в контитулах

\begin{document}
	
	\maketitle % Титульник
	\thispagestyle{empty} % Отсутствие номера страницы на титульной странице
	\clearpage % Делаем пустой титульник чтобы перенести оглавление
	\setcounter{page}{1} % Сброс счетчика страниц
	\thispagestyle{empty}
	\tableofcontents\label{sec:toc}
	\clearpage

	\section{Физические основы работы полупроводниковых приборов}
	
	\subsection{Атом}
	
	В основе электроники лежит использование потоков электронов, а иногда и других элементарных частиц, движущихся в газах или полупроводниках.
	
	\par Самая маленькая частица вещества, сохраняющая все его химические свойства, называется \textbf{молекулой}. 
	
	\par Каждый отдельный \textbf{атом} состоит из \textbf{ядра} и некоторого количества \textbf{электронов} ($\boldsymbol{e}$ - электрический заряд равен -1 в единицах элементарного заряда).
	
	\par \textbf{Атомные ядра} состоят из \textit{протонов} ($\boldsymbol{p}$ - заряд +1) и \textit{нейтронов} ($\boldsymbol{n}$ - заряд 0), которое "склеены" между собой с помощью \textit{глюонов} (квантов сильного взаимодействия).
	
	\begin{figure}[h]
		\centering
		\includegraphics[height=6cm]{img/2} 
		\captionsetup{font=footnotesize}
		\caption{Боровска модель атома.} 
	\end{figure}
	
	
	\par Электрон в атоме может двигаться только по определённым стационарным орбитам. Каждой орбите соответствует определённое количество электронов: 
	\\
	Для оболочки K: $1^{2} \times 2 = 2$. \\
	Для оболочки L: $2^{2} \times 2 = 8$. \\
	Для оболочки M: $3^{2} \times 2 = 18$. \\
	Для оболочки N: $4^{2} \times 2 = 32$. \\
	Для оболочки O: $5^{2} \times 2 = 50$. \\
	Для оболочки P: $6^{2} \times 2 = 72$. \\
	Для оболочки Q: $7^{2} \times 2 = 98$.
	
	\par Получается что в общей сложности 280 электронов. Значит должны быть атомы, содержащие такое количество электронов. Но, нет, потому что если оболочки K, L, M и N действительно могут иметь указанное мною количество электронов, то на оболочке O, на практике, их бывает не больше 18, на оболочке P - больше 32 и на оболочке Q больше 10. 
	
	\clearpage
 
    \subsection{Ионизация}
    
	\par Атомы могут терять свои электроны, этот процесс называется \textbf{ионизация}. Атомы, имеющие некомплектное количество электронов, носят название \textbf{ионов}. Атомы, у которых недостаёт одного или нескольких электронов, являются \textit{положительными ионами}, атомы с избытком электронов - \textit{отрицательными ионами}. Однако такие потери или приобретения электронов могут иметь место главным образом на внешней оболочке, то есть там, где меньше проявляется сила притяжения ядра. Именно эта оболочка определяет химические свойства элементов. 
	
    \par Внешнюю оболочку часто называют \textit{валентной}, имея в виду, что количество электронов на ней показывает, в какие комбинации атом может войти.
    
    \par \textit{Валентным числом} называют количество недостающих до стабильного состояния электронов или же, наоборот, количество электронов, которые атом способен отдать другому атому, чтобы стать стабильным. 
	
	\par Ионизация атомов может происходить от различных причин:
	\\
	1) высокие температуры;\\
	2) ультрафиолетовые лучи; и т.д. ....\\
	В электронных приборах чаще всего приходится приходится встречаться с ионизацией, возникающей в результате столкновения атома с какой-либо элементарной частицей, например с электроном. Если какая-то частица приближается к атому с большой скоростью(причём непосредственно соприкосновения между ними может и не произойти), то она может выбить из него один или несколько электронов и, следовательно, ионизировать его. При ударах меньшей силы атом не ионизируется, а полученная им энергия сообщается одному из его электронов, заставляя перейти его на более удалённую от ядра оболочку, где электроны обладают большей энергией. Однако очень скоро электрон возвращается на свою оболочку, отдавая при этом избыток энергии в виде излучения. При переходе электронов, находящихся в наиболее близких к ядру оболочках происходит излучение электромагнитных колебаний, соответствующих по частоте рентгеновским лучам. При переходе электронов с более удалённых от ядра оболочек атом излучает кванты энергии(фотоны).
	
	\par У различных веществ электроны не одинаково прочно удерживаются в системе атома.
	\\ 
	Особенно легко "теряют" свои электроны атомы металлов. Строение металлов представляют собой объёмную решётку, составленную из атомов. В промежутках между атомами этой решётки беспорядочно движутся во всех направлениях огромными количествами электронов, оторвавшихся от атомов. Эти электроны называются свободными. Скорость беспорядочного хаотичного движения электронов зависит от температуры металла. При повышении температуры она увеличивается. В любом куске металла свободных электронов столько, сколько их нужно для полного уравновешивания зарядов положительных ионов - атомов с неполным "комплектом" электронов. Поэтому в целом такой кусок металла не имеет заряда.
	
	\par Все явления электрического тока связаны с перемещением электрических зарядов, т.е. в большинстве случаев электронов или ионов. Любое перемещение даже одиночного электрона или иона является электрическим током, но практически движущиеся заряды проявляют свойства электрического тока только тогда, когда преобладающее количество имеющихся в данном объёме электрических зарядов движется в одном определённом направлении(упорядоченное движение).Хаотическое движение зарядов, например тепловое движение электронов в металле, не проявляет себя как электрический ток, потому что при подобном движении любому количеству электронов, перемещающихся в каком либо направлении, всегда противопоставляется такое же количество электронов, движущихся в обратном направлении, и их действие взаимно нейтрализуется.
	
	\par \textit{Все вещества в электрическом отношении делятся на проводники, диэлектрики и полупроводники}.
	\\
	Если атомы данного вещества легко теряют электроны и эти свободные электроны имеют возможность без особых затруднений перемещаться внутри этого вещества, то такое вещество называется \textit{проводником} электрического тока. Лучшим проводником являются металлы. В металлах всегда имеется огромное количество свободных электронов.
	\\
	Свободных электронов в \textit{диэлектриках} (непроводниках) практически нет, продвижение электронов в них крайне затруднительно. К диэлектрикам относятся, например, стекло, мрамор, слюда и т.д. 
	\\
	\textit{Полупроводники} занимают промежуточное место между проводниками и непроводниками. Их сопротивление больше, чем у проводников, и меньше, чем у диэлектриков. Кроме того, сопротивление полупроводников может сильно изменяться от разных причин - температуры, примесей и прочих. Наиболее распространённые полупроводники - кремний, германий, селен и другие.
	
	\begin{figure}[h]
		\centering
		\includegraphics[height=8cm]{img/3} 
		\captionsetup{font=footnotesize}
		\caption{Распределение электронов(общее количество которых носит название "атомного номера") по различным оболочкам у основных элементов, используемых для изготовления транзисторов. Жирным шрифтом обозначены цифры, определяющие валентность.} 
	\end{figure}
	
	\clearpage

	\subsection{Зонная теория}
	
	\par Согласно постулатам Бора в изолированном атоме электроны способны занимать лишь дискретные энергетические уровни, определяемые силами притяжения к ядру и силами отталкивания от других электронов. В твердом теле атомы расположены настолько близко друг к другу, что между ними возникают новые силы взаимодействия – это силы отталкивания между ядрами и между электронами соседних атомов и силы притяжения между всеми ядрами и всеми электронами. Под действием этих сил энергетические состояния в атомах изменяются: энергия одних электронов увеличивается, других – уменьшается. 
	\par В результате из каждого дискретного энергетического уровня атома или молекулы образуется энергетическая зона, состоящая из очень близко расположенных энергетических уровней (рис. 3). 

	\begin{figure}[h]
		\centering
		\includegraphics[height=8cm]{img/4} 
		\captionsetup{font=footnotesize}
		\caption{Упрощённая зонная диаграмма для проводников, полупроводников и диэлектриков.} 
	\end{figure}
	
	\par Разрешенная зона, в которой при температуре абсолютного нуля все энергетические зоны заняты электронами, называется \textbf{валентной}.
	
	\par Разрешенная зона, в которой при температуре абсолютного нуля электроны отсутствуют, называется \textbf{зоной проводимости}.
	
	\par \textbf{Запрещённая зона} - это разрыв между энергетическими уровнями, на котором электроны не могут находиться.
	
	\par \textbf{Под уровнем Ферми} понимается такой энергетический уровень, вероятность заполнения которого электронами равна половине.
	
	\par Для перехода электрона из низшей энергетической зоны в высшую требуется затратить энергию, равную ширине запрещенной зоны. При ширине запрещённой зоны в несколько электрон-вольт внешнее электрическое поле практически не может перевести электрон из валентной зоны в зону проводимости, так как энергия, приобретаемая электроном, движущимся ускоренно на длине свободного пробега, недостаточна для преодоления свободной зоны. \textit{Длинной свободного пробега} является расстояние, проходимое электроном между двумя соударениями с атомами кристаллической решётки. 
    
    \par Таким образом, способность твёрдого тела проводить ток под действием электрического поля зависит от структуры энергетических зон и степени их заполнения электронами.
    
	Расположение зон в материалах разных типов:
	\\
	1)\textit{Проводники} - зона проводимости и валентная зона перекрываются, образуя одну зону, называемую зоной перекрытия, таким образом, электрон может свободно перемещаться между ними, получив любую допустимо малую энергию. Таким образом, при приложении к телу разности потенциалов, электроны свободно движутся из точки с меньшим потенциалом в точку с большим, образуя электрический ток. К проводникам относят все металлы;
	\\
	2)\textit{Полупроводники} - зоны не перекрываются, и расстояние между ними (ширина запрещённой зоны) составляет менее 3,0 эВ. При абсолютном нуле температуры в зоне проводимости нет электронов, а валентная зона полностью заполнена электронами, которые не могут изменить своё квантовомеханическое состояние, то есть не могут упорядоченно двигаться при приложении электрического поля. Поэтому при нулевой абсолютной температуре собственные полупроводники не проводят электрический ток. При повышении температуры за счет теплового движения часть электронов, нарастающая при повышении температуры, «забрасывается» из валентной зоны в зону проводимости и собственный полупроводник становится электропроводным, причём его проводимость нарастает при увеличении температуры, так как растёт концентрация носителей заряда — электронов в зоне проводимости и дырок в валентной зоне. У полупроводников ширина запрещённой зоны относительно невелика, поэтому для перевода электронов из валентной зоны в зону проводимости требуется энергия меньшая, чем для диэлектрика, именно поэтому чистые (собственные, нелегированные) полупроводники обладают заметной проводимостью при ненулевой температуре; 
	\\
	3)\textit{Диэлектрики} - зоны как и у полупроводников не перекрываются, и расстояние между ними составляет, условно, более 3,0 эВ. Таким образом, для того, чтобы перевести электрон из валентной зоны в зону проводимости требуется значительная энергия (температура), поэтому диэлектрики ток при невысоких температурах практически не проводят.
	
	\par При достаточно высоких температурах все диэлектрики приобретают полупроводниковый механизм электропроводности. Отнесение вещества к тому или иному классу больше зависит от способа использования или предмета изучения вещества тем или иным автором. Иногда в классе полупроводников выделяют подкласс узкозонных полупроводников — с шириной запрещённой зоны менее 1 эВ.
	
	
	\clearpage

	\subsection{Собственная электропроводность полупроводников}
	
	\par Нам интересны \textit{периферийные электроны}(электроны на внешней оболочке), так как они легче отрываются от атома, потому что слабее притягиваются ядром. Электроны отрываются тогда, когда на внешней оболочке их мало - один, два или три. Золото, серебро и медь имеют всего лишь по одному периферийному электрону; железо, магний, цинк - по два, а алюминий даже три электрона. Эти электроны легко отрываются от атома и, став свободными, образуют тот поток электронов, который мы называем электрическим током.
	
	\par Рассмотрим строение полупроводникового материала, получившего наиболее широкое распространение в современной электронике, - кремния (Si). В кристалле этого полупроводника атомы располагаются в узлах кристаллической решетки, а электроны наружной электронной оболочки образуют устойчивые ковалентные связи, когда каждая пара валентных электронов принадлежит одновременно двум соседним атомам и образует связывающую эти атомы силу. Так как у элементов IV группы на наружной электронной оболочке располагаются по четыре валентных электрона, то в идеальном кристалле полупроводника все ковалентные связи заполнены, и все электроны прочно связаны со своими атомами
	
    \par  При температуре абсолютного нуля (T=0° K) все энергетические состояния внутренних зон и валентная зона занята электронами полностью, а зона проводимости совершенно пуста. Поэтому в этих условиях кристалл полупроводника является практически диэлектриком (рис. 4).
	
	\begin{figure}[h]
		\centering
		\includegraphics[height=5cm]{img/5} 
		\captionsetup{font=footnotesize}
		\caption{Структура связей атома кремния в кристаллической решётке.} 
	\end{figure}
	
	\par При температуре (T>0° K) в результате увеличения амплитуды тепловых колебаний атомов в узлах кристаллической решетки дополнительной энергии, поглощенной каким-либо электроном, может оказаться достаточно для разрыва ковалентной связи и перехода в зону проводимости, где электрон становится свободным носителем электрического заряда (рис. 5).
	
	\begin{figure}[h]
		\centering
		\includegraphics[height=8cm]{img/6} 
		\captionsetup{font=footnotesize}
		\caption{Генерация пары свободных носителей заряда "электрон - дырка".} 
	\end{figure}
	
	\par Электроны хаотически движутся внутри кристаллической решетки и представляют собой так называемый электронный газ. Электроны при своем движении сталкиваются с колеблющимися в узлах кристаллической решетки атомами, а в промежутках между столкновениями они движутся прямолинейно и равномерно.
    
    \par Одновременно с этим у того атома полупроводника, от которого отделился электрон, возникает незаполненный энергетический уровень в валентной зоне, называемый \textbf{дыркой}.
    
    \par Таким образом, в идеальном кристалле полупроводника при нагревании могут образовываться пары носителей электрических зарядов «электрон – дырка», которые обусловливают появление \textit{собственной электрической проводимости} полупроводника.
	
	\par Процесс образования пар «электрон – дырка» называют \textit{генерацией свободных носителей заряда}.
	
	\par После своего образования пары «электрон – дырка» существуют в течение некоторого времени, называемого \textit{временем жизни носителей} электрического заряда.
	
	\par В течение этого промежутка времени носители участвуют в тепловом движении, взаимодействуют с электрическими и магнитными полями, как единичные электрические заряды, перемещаются под действием градиента концентрации, а затем \textit{рекомбинируют}, т. е. электрон восстанавливает ковалентную связь.
	
	\par При рекомбинации электрона и дырки происходит высвобождение энергии, в зависимости от того, как расходуется эта энергия, рекомбинацию можно разделить на два вида: \textit{излучательную} и \textit{безызлучательную}.
	
	\par Излучательной является рекомбинация, при которой энергия, освобождающаяся при переходе электрона на более низкий энергетический уровень, излучается в виде кванта света – фотона.
	
	\par При безызлучательной рекомбинации избыточная энергия передается кристаллической решетке полупроводника, т. е. избыточная энергия идет на образование фононов – квантов тепловой энергии.
	
	\par Следует отметить, что генерация пар носителей заряда «электрон – дырка» и появление собственной электропроводности полупроводника может происходить не только под действием тепловой энергии, но и при любом другом способе энергетического воздействия на полупроводник: квантами лучистой энергии, ионизирующим излучением и т. д.

	\subsection{Распределение электронов по энергетическим уровням}
	
	\par При неизменном температурном состоянии полупроводника распределение электронов по	энергетическим уровням подчиняется квантовой статистике Ферми – Дирака. С ее помощью можно	определить концентрацию электронов в зоне проводимости, дырок в валентной зоне и определить зависимость удельной электропроводности полупроводника от температуры, наличия примесей и	других факторов.
	
    \par Вероятность заполнения электроном энергетического уровня W при температуре T определяется функцией распределения Ферми. 
    \\
    \begin{equation}
    f_n(W) = \frac{1}{1+e^{\frac{W-W_f}{kT}}}
    \end{equation}
    
    \par Соответственно, вероятность того, что квантовое состояние с энергией W свободно от электрона, т.е. занято дыркой.
    
    \begin{equation}
    	f_p(W)= 1 - f_n(W) = \frac{1}{1+e^{\frac{W_f-W}{kT}}}
    \end{equation}
    \\
    \(T\) – температура в градусах Кельвина;\\ \(k\) – постоянная Больцмана; \\
    \(W_f\) – энергия уровня Ферми, вероятность заполнения которого равна 0,5 при T=0°K.
    
    \begin{figure}[h]
    	\centering
    	\includegraphics[height=8cm]{img/7} 
    	\captionsetup{font=footnotesize}
    	\caption{Распределение электронов по энергетическим уровням для чистого полупроводника.} 
    \end{figure}
    
    \par При T=0°K функция распределения Ферми имеет ступенчатый характер. Это означает, что все энергетические уровни, находящиеся выше уровня Ферми, свободны. При T>0°K увеличивается вероятность заполнения электроном энергетического уровня, расположенного выше уровня Ферми. Поэтому ступенчатый характер функции распределения сменяется на более плавный в сравнительно узкой области энергий, близких к \(W_f\) 0°K .

    \subsection{Примесная электропроводность полупроводников}
    
    \par Электропроводность полупроводника может обусловливаться не только генерацией пар носителей «электрон – дырка» вследствие какого-либо энергетического воздействия, но и введением в структуру полупроводника определенных примесей. Примеси могут быть \textit{донорного} и \textit{акцепторного} типа.
    
    \par \textit{Легирование} - процесс введения примесей в полупроводник. Для легирования используется 3-х валентные бор, алюминий, индий, галлий и 5-и валентные сурьма, мышьяк, фосфор. Чем выше степень легирования, тем выше будет располагаться уровень Ферми у n - полупроводников.
    \par Полупроводники \(n\)-типа, у которых уровень Ферми располагается в зоне проводимости, называются \textit{вырожденными полупроводниками}.
    
    \par Уровень Ферми у полупроводников \(p\)-типа располагается ниже запрещённой зоны.
   
    \subsubsection{Донорные примеси}
    
    \par \textit{Донор} - это примесный атом или дефект кристаллической решётки, создающий в запрещённой зоне энергетический уровень, занятый в невозбуждённом состоянии электроном и способным в возбуждённом состоянии отдать электрон в зону проводимости.
    
    \begin{figure}[h]
    	\centering
    	\includegraphics[height=8cm]{img/8} 
    	\captionsetup{font=footnotesize}
    	\caption{Структура полупроводника с донорными примесями.} 
    \end{figure}
    
    \par При введении в 4-валентный полупроводник примесных 5-валентных атомов, например, сурьмы (Sb) атомы примесей замещают основные атомы в узлах кристаллической решетки. Четыре электрона атома примеси вступают в связь с четырьмя валентными электронами соседних атомов основного полупроводника. Пятый валентный электрон слабо связан со своим атомом и при сообщении ему незначительной энергии, называемой энергией активации, отрывается от атома и становится свободным. Примеси, увеличивающие число свободных электронов, называют донорными или просто \textit{донорами}. Малая энергия активации примесей (0,01.. .0,2 эВ) уже при комнатной температуре приводит к полной ионизации 5-валентных атомов примесей и появлению свободных электронов. Поскольку в этом случае появление свободных электронов не сопровождается одновременным увеличением количества дырок (ионизированные 5-валентные атомы, хотя и являются носителями положительного заряда, не могут свободно перемещаться по кристаллу или обмениваться валентными электронами с соседними атомами основного вещества), в таком полупроводнике концентрация электронов оказывается значительно больше концентрации дырок (дырки образуются только в результате разрыва ковалентных связей между атомами основного вещества). 
    
    \par Полупроводники, в которых концентрация свободных электронов превышает концентрацию дырок, называются полупроводниками с электронной электропроводностью или \textbf{полупроводниками n-типа} (негативная, отрицательная проводимость).
	
	\par Подвижные носители заряда, преобладающие в полупроводнике, называют основными. Соответственно те носители заряда, которые находятся в меньшем количестве, называются неосновными для данного типа полупроводника. \textit{В полупроводнике n-типа основными носителями заряда являются электроны, а неосновными - дырки}.
	
	\begin{figure}[h]
		\centering
		\includegraphics[height=8cm]{img/9} 
		\captionsetup{font=footnotesize}
		\caption{Зонная диаграмма(а) и распределение электронов по энергетическим уровням (б) полупроводника с донорскими примесями} 
	\end{figure}
	
	\par В отличие от идеальных, чистых полупроводников диаграмма распределения электронов по энергетическим уровням в полупроводниках \(n\)-типа изменяется (рис. 8). Уровень Ферми в этом случае будет смещаться вверх, к границе зоны проводимости \(W_п\), так как малейшее приращение энергии приводит к его переходу в зону проводимости.
	
	\subsubsection{Акцепторные примеси}
	
	\par \textit{Акцептор} - это примесный атом или дефект кристаллической решётки, создающий в запрещённой зоне энергетический уровень, свободный от электрона в невозбуждённом состоянии и способный захватить электрон из валентной зоны в возбуждённом состоянии.
	
	\begin{figure}[h]
		\centering
		\includegraphics[height=8cm]{img/10} 
		\captionsetup{font=footnotesize}
		\caption{Структура полупроводника с акцепторными примесями} 
	\end{figure}
	
	\par Если в кристалле 4-х валентного элемента часть атомов замещена атомами 3-х валентного элемента, например, индия (In), то для образования четырех ковалентных связей у примесного атома не хватает одного электрона. Этот электрон может быть получен от атома основного элемента полупроводника за счет разрыва ковалентной связи. Разрыв связи приводит к появлению дырки. Примеси, захватывающие валентные электроны, называют акцепторными или \textit{акцепторами}. Ввиду малого значения энергии активации акцепторов уже при комнатной температуре многие валентные электроны переходят на уровни акцепторов. Эти электроны, превращая примесные атомы в отрицательные ионы, теряют способность перемещаться по кристаллической решетке, а образовавшиеся при этом дырки могут участвовать в создании электрического тока. За счет ионизации атомов исходного материала часть валентных электронов становится свободной. Однако свободных электронов значительно меньше чем дырок.
	
	\par Полупроводники в которых концентрация дырок превышает концентрацию свободных электронов, называются полупроводниками с дырочной электропроводностью или \textbf{полупроводниками p-типа}(позитивный, положительный тип проводимости). \textit{Дырки в таких полупроводниках являются основными, а электроны - неосновными.}
	
	\begin{figure}[h]
		\centering
		\includegraphics[height=8cm]{img/11} 
		\captionsetup{font=footnotesize}
		\caption{Зонная диаграмма(а) и распределение электронов по энергетическим уровням (б) полупроводника с акцепторными примесями} 
	\end{figure}
	
	\par Вероятность захвата электрона и перехода его в валентную зону возрастает в полупроводниках p-типа, поэтому уровень Ферми здесь смещается вниз, к границе валентной зоны (рис. 10).
	\par Следует отметить, что \textit{при очень больших концентрациях примесей} в полупроводниках уровень Ферми может даже выходить за пределы запрещённой зоны либо в зону проводимости (в полупроводниках \(n\)-типа) либо в зону валентную (в полупроводниках \(p\)-типа). Такие полупроводники называются \textbf{вырожденными}

	\subsection{Процессы переноса зарядов в полупроводниках}
	
	\par В полупроводниках процесс переноса зарядов может наблюдаться при наличии электронов в зоне проводимости и при неполном заполнении электронами валентной зоны. При выполнении данных условий и при отсутствии градиента температуры(распределение температуры меняется вдоль пространства) перенос носителей зарядов возможен либо под действием электрического поля, либо под действием градиента концентрации(в разных частях полупроводника различная концентрация электронов и дырок, что приводит к их диффузии в направлении с более высокой концентрации к более низкой) носителей заряда.
	
	\subsubsection{Дрейфовый ток}
	
	\par \textit{Дрейфом} называют направленное движение носителей заряда под действием электрического поля.
	
	\par Электроны, получая ускорение в электрическом поле приобретают на средней длине свободного пробега добавочную составляющую скорости, которая называется \textit{дрейфовой скоростью} \(V_n\), к своей средней скорости движения.
	
	\par Дрейфовая скорость электрона мала по сравнению со средней скоростью их теплового движения в обычных условиях. Плотность дрейфового тока:
	
	\begin{equation}
	    J_n = qnV_n
	\end{equation}
	
	\(n\) - \(\text{концентрация электронов в 1 см}^3\);
	\par \(q\) - заряд электрона.\\
	
	\par Дрейфовая скорость, приобретаемая электроном в поле единичной напряжённости $E = 1 \, \text{В/м}$, называется \textit{подвижностью}:
	
	\begin{equation}
		\mu = \dfrac{V_n}{E}
	\end{equation}
	
	\par Поэтому плотность дрейфового тока электронов:
	
	\begin{equation}
		J_n = qn\mu E
	\end{equation}
	
	\par Составляющая электрического тока под действием внешнего электрического поля называется \textit{дрейфовым током}. Полная плотность дрейфового тока при наличии свободных электронов и дырок равна сумме электронной и дырочной составляющих:
	
	\begin{equation}
		J = J_n + J_p = qE(n\mu_n + p\mu_p)
	\end{equation}
	
	\par \(E\) - напряжённость приложенного электрического поля.
	
    \par \textit{Удельная электрическая проводимость} $\sigma$ равна отношению плотности дрейфового тока к величине напряжённости электрического поля \(E\), вызвавшего этот ток:  
    
    \begin{equation}
    	\sigma = \dfrac{J}{E}
    \end{equation}
    
    \par То есть электропроводность твёрдого тела зависит от концентрации носителей электрического заряда \(n\) и от подвижности \(\mu\).
    
    \subsubsection{Диффузия носителей заряда}
    
    \par При неравномерном распределении концентрации носителей заряда в объёме полупроводника и отсутствии градиента температуры происходит \textit{диффузия} - движение носителей заряда из-за градиента концентрации, т.е. происходит выравнивание концентрации носителей заряда по объёму полупроводника(такую диффузию не следует путать с химической диффузией, так как в этом случае атомы не перемещаются, а перенос происходит только носителей заряда).
    
    \par При отсутствии внешнего электрического поля и при условии динамического равновесия в кристалле полупроводника устанавливается единый уровень Ферми для обеих областей проводимости.
    
    \clearpage

	\subsection{Электрические переходы}
	
	\par \textit{Электрическим переходом} в полупроводнике называется граничный слой между двумя областями, физические характеристики которых имеют существенные физические различия.
	
	\par Различают следующие виды электрических переходов:
	
	\par 1) \textit{электронно-дырочный}, или \(p-n\)-переход -- переход между двумя областями полупроводника, имеющим разный тип проводимости;
	
	\par 2) переходы между двумя областями, если одна из них является металлом, а другая полупроводником \(p-\) или \(n-\)типа(\textit{переход метал - полупроводник});
	
	\par 3) переходы между двумя областями с одним типом электропроводности, отличающиеся значением концентрации примесей;
	
	\par 4) переходы между двумя полупроводниковыми материалами с различной шириной запрещённой зоны(\textit{гетеропереходы}).
	
	\subsubsection{Электронно-дырочный переход}
	
	\par Граница между двумя областями монокристалла полупроводника, одна из которых имеет электропроводность типа \(p\), а другая - типа \(n\) называется электронно-дырочным переходом. Концентрации основных носителей заряда в областях \(p\) и \(n\) могут быть равными или существенно отличаться.
	
	\par \(P-n\) переход, у которого концентрация дырок и электронов практически равны называют \textit{симметричным}. Если концентрации основных носителей заряда различны и отличаются в 100...1000 раз, то такие переходы называют \textit{несимметричными}.
	\par Несимметричные \(p-n\)-переходы используются шире, чем симметричные, поэтому в дальнейшем будем рассматривать только их.
	
	\par Рассмотрим монокристалл полупроводника (рис.11), в котором, с одной стороны, введена, акцепторная примесь, обусловившая возникновение здесь электропроводности типа \(p\), а с другой стороны, введена донорная примесь, благодаря которой там возникла электропроводность типа \(n\). 
	
	\begin{figure}[h]
		\centering
		\includegraphics[height=6cm]{img/12} 
		\captionsetup{font=footnotesize}
		\caption{Начальный момент образования \(p-n\)-перехода} 
	\end{figure}
	
	\par Каждому подвижному носителю заряда в области \(p\) (дырке) соответствует отрицательно заряженный ион акцепторной примеси, но неподвижный, находящийся в узле кристаллической решётке, а в области \(n\) каждому свободному электрону соответствует положительно заряженный ион донорной примеси, в результате чего весь монокристалл остаётся электрически нейтральным.
	
	\par Свободные носители электрических зарядов под действием градиента концентрации начинают перемещаться из мест с большей концентрацией в места с меньшей концентрацией. Так, дырки будут диффундировать из области \(p\) в область \(n\), а электроны, наоборот, из области \(n\) в область \(p\). Это направленное навстречу друг другу перемещение электрических зарядов образует диффузионный ток \(p-n\)-перехода.
	
	\begin{figure}[h]
		\centering
		\includegraphics[height=8cm]{img/13} 
		\captionsetup{font=footnotesize}
		\caption{\(P-n\)-переход при отсутствии внешнего напряжения} 
	\end{figure} 
	
	\par Но как только дырка перейдёт из области \(p\) в область \(n\), она оказывается в окружении электронов, являющихся основными носителями электрических зарядов в области \(n\). Поэтому велика вероятность того, что какой-либо электрон заполнит свободный уровень в дырке и произойдёт явление \textbf{рекомбинации}(при рекомбинации выделяется квант энергии, например света), в результате которой не будет ни дырки, ни электрона, а \textit{останется электрически нейтральный атом полупроводника}. 
	
	\par Но если раньше положительный электрический заряд каждой дырки компенсировался отрицательным зарядом иона акцепторной примеси в области \(p\), а заряд электрона - положительным зарядом иона донорной примеси в области \(n\), то после рекомбинации дырки и электрона электрические заряды \textit{неподвижных ионов примесей}, породивших эту дырку и электрон, остались не скомпенсированными. И в первую очередь не скомпенсированные заряды ионов примесей проявляют себя вблизи границы раздела(рис.12), где образуется слой пространственных зарядов, разделённых узким промежутком  $\delta$. Между этими зарядами возникает электрическое поле с напряжённостью \(E\), которое называют \textit{полем потенциального барьера}, а разность потенциалов на границе раздела двух зон, обуславливающих это поле, называют \textit{контактной разницей потенциалов} $\Delta \varphi_K$.
	
	\par Это электрическое поле начинает действовать на подвижные носители электрических зарядов. Так, дырки в области \(p\) - основные носители, попадая в зону действия этого поля, испытывают со стороны него тормозящее, отталкивающее действие и, перемещаясь вдоль силовых линий этого поля, будут вытолкнуты в глубь области \(p\). Аналогично, электроны из области \(n\), попадая в зону действия поля потенциального барьера, будут вытолкнуты им вглубь области \(n\). Таким образом, в узкой области $\delta$, где действует поле потенциального барьера, образуется слой, где практически отсутствуют свободные носители электрических зарядов и вследствие этого обладающий высоким сопротивлением. Это так называемый \textbf{запирающий слой}.
	
	\par Если же в области \(p\) вблизи границы раздела каким-либо образом окажется свободный электрон, являющийся неосновным носителем для этой области, то он со стороны электрического поля потенциального барьера будет испытывать ускоряющее воздействие, вследствие чего этот электрон будет переброшен через границу раздела в область \(n\), где он будет являться основным носителем. Аналогично, если в области \(n\) появится неосновной носитель – дырка, то под действием поля потенциального барьера она будет переброшена в область \(p\) , где она будет уже основным носителем. Движение неосновных носителей через \(p-n\)-переход под действием электрического поля потенциального барьера обусловливает составляющую дрейфового тока.
	
	\par При отсутствии внешнего электрического поля устанавливается динамическое равновесие между потоками основных и неосновных носителей электрических зарядов, то есть между диффузионной и дрейфовой составляющими тока \(p-n\)-перехода, поскольку эти составляющие направлены навстречу друг другу.
	
	\par Потенциальная диаграмма \(p-n\)-перехода изображена на рис.12, причём за нулевой потенциал принят потенциал на границе раздела областей.
	
	\begin{figure}[h]
		\centering
		\includegraphics[height=6cm]{img/14} 
		\captionsetup{font=footnotesize}
		\caption{Зонная диаграмма \(p-n\)-перехода, иллюстрирующая баланс токов в равновесном состоянии} 
	\end{figure}
	
	\par  Однако, поскольку в полупроводниках \(p\)-типа уровень Ферми смещается к потолку валентной зоны \(W\)в\(p\) , а в полупроводниках \(n\)-типа – ко дну зоны проводимости \(W\)п\(n\), то на ширине \(p-n\)-перехода $\delta$ диаграмма энергетических зон (рис.13) искривляется и образуется потенциальный барьер:
	
	\begin{equation}
		\Delta \varphi_K = \dfrac{\Delta W}{q}
	\end{equation}
	
	\par $\Delta W$ - энергетический барьер, который необходимо преодолеть электрону в области \(n\), чтобы он мог перейти в области \(p\), или аналогично для дырки в области \(p\), чтобы она могла перейти в область \(n\).
	
    \par Высота потенциального барьера зависит от концентрации примесей, так как при ее изменении изменяется уровень Ферми, смещаясь от середины запрещенной зоны к верхней или нижней ее границе.
    
    \subsubsection{Вентильное свойство $\boldsymbol{p-n}$-перехода}
    
    \par \(P-n\)-переход, обладает свойством изменять свое электрическое сопротивление в зависимости от направления протекающего через него тока. Это свойство называется \textbf{вентильным}, а прибор, обладающий таким свойством, называется \textit{электрическим вентилем}.
    
    \par Рассмотрим с, к которому подключен внешний источник напряжения \(U\)вн с полярностью, указанной на (рис.14), «+» к области \(p\)-типа, «–» к области \(n\)-типа. Такое подключение называют \textit{прямым включением \(p-n\)-перехода}.
    
    \begin{figure}[h]
    	\centering
    	\includegraphics[height=10cm]{img/15} 
    	\captionsetup{font=footnotesize}
    	\caption{Прямое смещение \(p-n\)-перехода} 
    \end{figure}
    
    \par Тогда напряжённость электрического поля внешнего источника \(E\)вн будет направлена навстречу напряжённости поля потенциального барьера \(E\) и, следовательно, приведёт к снижению результирующей напряжённости \(E\)рез:
    
    \begin{equation}
    	   \text{E рез} = E - \text{E вн}
    \end{equation}
    
    \par Это приведет, в свою очередь, к снижению высоты потенциального барьера и увеличению количества основных носителей, диффундирующих через границу раздела в соседнюю область, которые образуют так называемый \textbf{прямой ток $\boldsymbol{p-n}$-перехода}. При этом вследствие уменьшения тормозящего, отталкивающего действия поля потенциального барьера на основные носители, ширина запирающего слоя $\delta$ уменьшается $(\delta^{'} <  \delta)$ и, соответственно, уменьшается его сопротивление.
    
    \par По мере увеличения внешнего напряжения прямой ток \(p-n\)-перехода возрастает. \textit{Основные носители после перехода границы раздела становятся неосновными в противоположной области полупроводника} и, углубившись в нее, рекомбинируют с основными носителями этой области, но, пока подключен внешний источник, ток через переход поддерживается непрерывным поступлением электронов из внешней цепи в \(n\)-область и уходом их из \(p\)-области во внешнюю цепь, благодаря чему восстанавливается концентрация дырок в \(p\)-области.
    
    \begin{figure}[h]
    	\centering
    	\includegraphics[height=6cm]{img/16} 
    	\captionsetup{font=footnotesize}
    	\caption{Зонная диаграмма прямого смещения \(p-n\)-перехода, иллюстрирующая дисбаланс токов} 
    \end{figure}
    
    \par Введение носителей заряда через \(p-n\)-переход при понижении высоты потенциального барьера в область полупроводника, где эти носители являются неосновными, называют \textbf{инжекцией носителей заряда}. 
    \par При протекании прямого тока из дырочной области \(p\) в электронную область \(n\) инжектируются дырки, а из электронной области в дырочную – электроны.
    
    \par Инжектирующий слой с относительно малым удельным сопротивлением называют \textbf{эмиттером}; слой, в который происходит инжекция неосновных для него носителей заряда, – \textbf{базой}.
    \\
    
    \par Если к \(p-n\)-переходу подключить внешний источник с противоположной полярностью «–» к области \(p\)-типа, «+» к области \(n\)-типа (рис.16), то такое подключение называют обратным включением \(p-n\)-перехода (или обратным смещением \(p-n\)-перехода).
    
    \begin{figure}[h]
    	\centering
    	\includegraphics[height=10cm]{img/17} 
    	\captionsetup{font=footnotesize}
    	\caption{Обратное смещение \(p-n\)-перехода} 
    \end{figure}
    
     \par В данном случае напряженность электрического поля этого источника \(E\)вн будет направлена в ту же сторону, что и напряженность электрического поля \(E\) потенциального барьера; высота потенциального барьера возрастает, а ток диффузии основных носителей практически становится равным нулю. Из-за усиления тормозящего, отталкивающего действия суммарного электрического поля на основные носители заряда ширина запирающего слоя $\delta$ увеличивается $(\delta^{''} > \delta)$, а его сопротивление резко возрастает.
     
     \begin{figure}[h]
     	\centering
     	\includegraphics[height=6cm]{img/18} 
     	\captionsetup{font=footnotesize}
     	\caption{Зонная диаграмма обратного смещения \(p-n\)-перехода, иллюстрирующая дисбаланс токов} 
     \end{figure}
     
     \par Теперь через \(p-n\)-переход будет протекать очень маленький ток, обусловленный перебросом суммарным электрическим полем на границе раздела, неосновных носителей, возникающих под действием различных ионизирующих факторов, в основном теплового характера. Процесс переброса неосновных носителей заряда называется \textbf{экстракцией}. Этот ток имеет дрейфовую природу и называется \textbf{обратным током $\boldsymbol{p-n}$-перехода}.
     
     \par \textbf{Выводы:}
      
     \par 1) \(p-n\)-переход образуется на границе \(p-\) и \(n-\)областей, созданных в монокристалле полупроводника. 
     
     \par 2) В результате диффузии в \(p-n\)-переходе возникает электрическое поле - потенциальный барьер, препятствующий выравниванию концентрации основных носителей заряда в соседних областях.
     
     \par 3) При отсутствии внешнего напряжения \(U\)вн в \(p-n\)-переходе устанавливается динамическое равновесие: диффузионный ток становится равным по величине дрейфовому току, образованному не основными носителями заряда, в результате чего ток через \(p-n\)-переход становится равным нулю. 
      
     \par 4) При прямом смещении \(p-n\)-перехода потенциальный понижается и через переход протекает относительно большой диффузный ток.
     
     \par 5) При обратном смещении \(p-n\)-перехода потенциальный барьер повышается, диффузионный ток уменьшается до нуля и через переход протекает малый по величине дрейфовый ток.
     
     \par Это означает, что \(p-n\)-переход обладает односторонней проводимостью. Из-за полевого потенциального барьера носители заряда могут двигаться только в определенном направлении через переход. Это приводит к созданию односторонней проводимости: электроны могут диффундировать только из области \(n\) к области \(p\), а дырки - из области \(p\) к области \(n\).
     
     \par 6) Ширина \(p-n\)-перехода(в этой области происходит рекомбинация основных носителей заряда (дырок и электронов) и формируется полевой потенциальный барьер) зависит: от концентраций примеси в \(p-\) и \(n-\)областях, от знака и величины приложенного внешнего напряжения \(U\)вн. При увеличении концентрации примесей ширина \(p-n\)-перехода уменьшается и наоборот. С увеличением прямого напряжения ширина \(p-n\)-перехода уменьшается. При увеличении обратного напряжения ширина \(p-n\)-перехода увеличивается. 
     
	\clearpage

	\section{Основные понятия об электрической цепи}
	
	
	\begin{definition}
		\textbf{Электрической цепью} называют совокупность гальванически соединённых(связаны между собой проводом) друг с другом источников электрической энергии и её потребителей(нагрузок), в которых может возникать электрический ток.
	\end{definition}
	
	\begin{definition}
		\textbf{Источники} преобразуют тот или иной вид энергии(энергия сжигаемого топлива, падающей воды, атомная и химическая энергия и т.д.
	\end{definition}
	
	\begin{definition}
		\textbf{Приёмники}, наоборот, преобразуют электрическую энергию в другие его виды.
	\end{definition}
	
	\begin{definition}
		\textbf{Ветвь} — это участок цепи, состоящий из последовательно соединённых элементов и заключённых между двумя узлами. В каждой ветви существует свой ток. 
	\end{definition}
	
	\begin{definition}
		\textbf{Узел} — это точка в элементарной схеме цепи, где гальванически соединяются не менее трёх ветвей 
	\end{definition}
	
	\begin{definition}
		\textbf{Контур} — это любой замкнутый путь на схеме. 
	\end{definition}
	
	\textbf{Элементы электрической цепи делят на:}\\
	1)Активный —-- источники;\\
	2)Пассивные —-- потребители или приёмники
	
	\clearpage
	
	\subsection{Подраздел}
	Тут будет текст подраздела.
	\clearpage
	\section{Заключение}
	Здесь будет заключение.
	
	
	
\end{document}